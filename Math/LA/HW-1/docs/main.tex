\documentclass[a4paper,14pt]{article}
\usepackage{amsmath}
\usepackage[russian]{babel}
\usepackage{xcolor}
\usepackage[margin=1.3cm]{geometry}
\usepackage{enumitem}

\begin{document}

\large

\subsection*{\textcolor{blue}{Цель}}
Изучить и сделать анализ различных методов векторной алгебры, аналитической геометрии, сравнить эти методы для решения задачи на определение принадлежности точки тетраэдру. Рассмотреть варианты расположения этой точки: внутри, снаружи, на грани, на ребре, совпадение с вершиной. Оценить эффективность и применимость каждого исследуемого метода.

\subsection*{\textcolor{blue}{Постановка задачи}}
Установить, принадлежит ли точка \(P(x, y, z)\) тетраэдра \(ABCD\), который определяется вершинами \(A, B, C, D\). Исследовать случаи:

\begin{enumerate}[nosep, leftmargin=*]
    \item точка находится внутри тетраэдра
    \item точка находится на грани тетраэдра
    \item точка находится на ребре тетраэдра
    \item точка совпадает с вершиной тетраэдра
    \item точка находится снаружи тетраэдра
\end{enumerate}

\subsection*{\textcolor{blue}{Вершины пирамиды}}
\noindent
\setlength{\tabcolsep}{3pt}
\begin{tabular}{|c|c|}
\hline
\textbf{Вершина} & \textbf{Координаты} \\
\hline
A & $(1,1,1)$ \\
\hline
B & $(2,2,0)$ \\
\hline
C & $(0,2,2)$ \\
\hline
D & $(2,0,2)$ \\
\hline
\end{tabular}

\vspace{1.0em}
\noindent Тетраэдр не является вырожденным т.к.: \\
$\overrightarrow{AB} = (1, 1, -1)$ \\
$\overrightarrow{AC} = (-1, 1, 1)$ \\
$\overrightarrow{AD} = (1, -1, 1)$

\noindent
$(\overrightarrow{AB} \times \overrightarrow{AC}) \cdot \overrightarrow{AD} = 
\begin{vmatrix}
1 & 1 & -1 \\
-1 & 1 & 1 \\
1 & -1 & 1 \\
\end{vmatrix} = 4$

\vspace{0.8em}
\noindent
\[
V_{ABCD} = \frac{1}{6} \left| (\overrightarrow{AB} \times \overrightarrow{AC}) \cdot \overrightarrow{AD} \right| 
= \frac{1}{6} \cdot |4| 
= \frac{2}{3} \neq 0 \quad \Rightarrow \quad \text{тетраэдр задан верно}
\]

\subsection*{\textcolor{blue}{Методы решения}}
\subsubsection*{\textcolor{blue}{Метод 1: Метод объёмов (через смешанное произведение)}}

\paragraph{Основная идея}
Вычисляем объём тетраэдра $ABCD$ и объёмы четырёх тетраэдров с вершиной в точке $P$: $PBCD$, $APCD$, $ABPD$, $ABCP$ с помощью смешанного произведения векторов. Сравниваем сумму объёмов $V_{PBCD} + V_{APCD} + V_{ABPD} + V_{ABCP}$ с объёмом $V_{ABCD}$. Каждый объём должен быть больше 0.

\paragraph{Описание математической реализации}
Метод основан на геометрическом свойстве: точка $P$ принадлежит тетраэдру $ABCD$ тогда и только тогда, когда сумма объёмов четырёх тетраэдров с общей вершиной $P$ равна объёму исходного тетраэдра.

\begin{enumerate}[leftmargin=*, itemsep=0.5em]


    \item \textbf{Точка внутри тетраэдра} \\
    Возьмём точку $P\left(\frac{5}{4}, \frac{5}{4}, \frac{5}{4}\right)$ (центроид)
    
    Вычисляем векторы:
    \begin{align*}
    \overrightarrow{AP} &= \left(\frac{1}{4},\ \frac{1}{4},\ \frac{1}{4}\right) \\
    \overrightarrow{PB} &= \left(\frac{3}{4},\ \frac{3}{4},\ -\frac{5}{4}\right) \\
    \overrightarrow{PC} &= \left(-\frac{5}{4},\ \frac{3}{4},\ \frac{3}{4}\right) \\
    \overrightarrow{PD} &= \left(\frac{3}{4},\ -\frac{5}{4},\ \frac{3}{4}\right) \\
    \overrightarrow{AB} &= \left(1,\ 1,\ -1\right) \\
    \overrightarrow{AC} &= \left(-1,\ 1,\ 1\right) \\
    \overrightarrow{AD} &= \left(1,\ -1,\ 1\right)
    \end{align*}
    
    Вычисляем объёмы:
    \begin{align*}
    V_{PBCD} &= \frac{1}{6} \left| 
    \begin{vmatrix}
    \frac{3}{4} & \frac{3}{4} & -\frac{5}{4} \\
    -\frac{5}{4} & \frac{3}{4} & \frac{3}{4} \\
    \frac{3}{4} & -\frac{5}{4} & \frac{3}{4}
    \end{vmatrix} \right| = \frac{1}{6} > 0 \\
    V_{APCD} &= \frac{1}{6} \left| 
    \begin{vmatrix}
    \frac{1}{4} & \frac{1}{4} & \frac{1}{4} \\
    -1 & 1 & 1 \\
    1 & -1 & 1
    \end{vmatrix} \right| = \frac{1}{6} > 0  \\
    V_{ABPD} &= \frac{1}{6} \left| 
    \begin{vmatrix}
    1 & 1 & -1 \\
    \frac{1}{4} & \frac{1}{4} & \frac{1}{4} \\
    1 & -1 & 1
    \end{vmatrix} \right| = \frac{1}{6} > 0  \\
    V_{ABCP} &= \frac{1}{6} \left| 
    \begin{vmatrix}
    1 & 1 & -1 \\
    -1 & 1 & 1 \\
    \frac{1}{4} & \frac{1}{4} & \frac{1}{4}
    \end{vmatrix} \right| = \frac{1}{6} > 0 
    \end{align*}
    
    Проверяем: $V_{PBCD} + V_{APCD} + V_{ABPD} + V_{ABCP} = \frac{1}{6} + \frac{1}{6} + \frac{1}{6} + \frac{1}{6} = \frac{4}{6} = \frac{2}{3} = V_{ABCD}$
    

    \item \textbf{Точка на грани (правильный пример)} \\
    Точка $P$ = $\alpha$$A$ + $\beta$$B$ + $\gamma$$C$ в грани $(ABC)$ \\
    $\alpha$ + $\beta$ + $\gamma$ = 1  \\  Тогда возьмём $\alpha$=0.4, $\beta$=0.3, $\gamma$=0.3 \\  $\alpha$ > 0,  $\beta$ > 0,  $\gamma$ > 0 \\
    $P(1, 1.6, 1)$ на грани $ABC$ (получена как $P = 0.4A + 0.3B + 0.3C$)
    
    Вычисляем векторы:
    \begin{align*}
    \overrightarrow{AP} &= (0,\ 0.6,\ 0) \\
    \overrightarrow{PB} &= (1,\ 0.4,\ -1) \\
    \overrightarrow{PC} &= (-1,\ 0.4,\ 1) \\
    \overrightarrow{PD} &= (1,\ -1.6,\ 1) \\
    \overrightarrow{AB} &= (1,\ 1,\ -1) \\
    \overrightarrow{AC} &= (-1,\ 1,\ 1) \\
    \overrightarrow{AD} &= (1,\ -1,\ 1)
    \end{align*}
    
    Вычисляем объёмы:
    \begin{align*}
    V_{PBCD} &= \frac{1}{6} \left| 
    \begin{vmatrix}
    1 & 0.4 & -1 \\
    -1 & 0.4 & 1 \\
    1 & -1.6 & 1
    \end{vmatrix} \right| > 0 \\
    V_{APCD} &= \frac{1}{6} \left| 
    \begin{vmatrix}
    0 & 0.6 & 0 \\
    -1 & 1 & 1 \\
    1 & -1 & 1
    \end{vmatrix} \right| > 0 \\
    V_{ABPD} &= \frac{1}{6} \left| 
    \begin{vmatrix}
    1 & 1 & -1 \\
    0 & 0.6 & 0 \\
    1 & -1 & 1
    \end{vmatrix} \right| > 0 \\
    V_{ABCP} &= \frac{1}{6} \left| 
    \begin{vmatrix}
    1 & 1 & -1 \\
    -1 & 1 & 1 \\
    0 & 0.6 & 0
    \end{vmatrix} \right| = 0 \quad \text{(точка в плоскости ABC)}
    \end{align*}

    Проверяем: $V_{PBCD} + V_{APCD} + V_{ABPD} + 0 = V_{ABCD}$





    

    \item \textbf{Точка на грани тетрадэдра} \\
    Точка $P$ = $\alpha$$A$ + $\beta$$B$ + $\gamma$$C$ в грани $(ABC)$ \\
    $\alpha$ + $\beta$ + $\gamma$ = 1  \\  Тогда $\alpha$=0.2, $\beta$=0.3, $\gamma$=0.5 \\  $\alpha$ > 0,  $\beta$ > 0,  $\gamma$ > 0
    

\end{enumerate}

\end{document}