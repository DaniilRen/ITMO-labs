\documentclass[a4paper,14pt]{article}
\usepackage{amsmath}
\usepackage[russian]{babel}
\usepackage{xcolor}
\usepackage[margin=1.3cm]{geometry}
\usepackage{enumitem}
\usepackage{listings}
\usepackage{graphicx}

% Настройки для вставки кода
\lstset{
    basicstyle=\ttfamily\footnotesize,
    breaklines=true,
    breakatwhitespace=true,
    frame=single,
    numbers=left,
    stepnumber=1,
    numbersep=5pt,
    showstringspaces=false,
    tabsize=4,
    xleftmargin=10pt,
    xrightmargin=10pt
}

\begin{document}

\large

\section*{\textcolor{blue}{Практическая работа: Векторы и аналитическая геометрия в прикладных задачах (Задача принадлежности точки тетраэдру)}}

\par\noindent\rule{\textwidth}{0.4pt}

\subsection*{\textcolor{blue}{Цель}}
Изучить и сделать анализ различных методов векторной алгебры, аналитической геометрии, сравнить эти методы для решения задачи на определение принадлежности точки тетраэдру. Рассмотреть варианты расположения этой точки: внутри, снаружи, на грани, на ребре, совпадение с вершиной. Оценить эффективность и применимость каждого исследуемого метода.

\subsection*{\textcolor{blue}{Постановка задачи}}
Установить, принадлежит ли точка \(P(x, y, z)\) тетраэдра \(ABCD\), который определяется вершинами \(A, B, C, D\). Исследовать случаи:

\begin{enumerate}[nosep, leftmargin=*]
    \item точка находится внутри тетраэдра
    \item точка находится на грани тетраэдра
    \item точка находится на ребре тетраэдра
    \item точка совпадает с вершиной тетраэдра
    \item точка находится снаружи тетраэдра
\end{enumerate}

\subsection*{\textcolor{blue}{Вершины пирамиды}}
\noindent
\setlength{\tabcolsep}{3pt}
\begin{tabular}{|c|c|}
\hline
\textbf{Вершина} & \textbf{Координаты} \\
\hline
A & $(1,1,1)$ \\
\hline
B & $(2,2,0)$ \\
\hline
C & $(0,2,2)$ \\
\hline
D & $(2,0,2)$ \\
\hline
\end{tabular}

\vspace{1.0em}
\noindent Тетраэдр не является вырожденным т.к.: \\
$\overrightarrow{AB} = (1, 1, -1)$ \\
$\overrightarrow{AC} = (-1, 1, 1)$ \\
$\overrightarrow{AD} = (1, -1, 1)$

\noindent
$(\overrightarrow{AB} \times \overrightarrow{AC}) \cdot \overrightarrow{AD} = 
\begin{vmatrix}
1 & 1 & -1 \\
-1 & 1 & 1 \\
1 & -1 & 1 \\
\end{vmatrix} = 4$

\vspace{0.8em}
\noindent
\[
V_{ABCD} = \frac{1}{6} \left| (\overrightarrow{AB} \times \overrightarrow{AC}) \cdot \overrightarrow{AD} \right| 
= \frac{1}{6} \cdot |4| 
= \frac{2}{3} \neq 0 \quad \Rightarrow \quad \text{тетраэдр задан верно}
\]

\subsection*{\textcolor{blue}{Методы решения}}
\subsubsection*{\textcolor{blue}{Метод 1: Метод объёмов (через смешанное произведение)}}

\paragraph{Основная идея}
Вычисляем объём тетраэдра $ABCD$ и объёмы четырёх тетраэдров с вершиной в точке $P$: $PBCD$, $APCD$, $ABPD$, $ABCP$ с помощью смешанного произведения векторов. Сравниваем сумму объёмов $V_{PBCD} + V_{APCD} + V_{ABPD} + V_{ABCP}$ с объёмом $V_{ABCD}$. Каждый объём должен быть больше 0.

\paragraph{Описание математической реализации}
Метод основан на геометрическом свойстве: точка $P$ принадлежит тетраэдру $ABCD$ тогда и только тогда, когда сумма объёмов четырёх тетраэдров с общей вершиной $P$ равна объёму исходного тетраэдра.

\begin{enumerate}[leftmargin=*, itemsep=0.5em]


    \item \textbf{Точка внутри тетраэдра} \\
    Возьмём точку $P\left(\frac{5}{4}, \frac{5}{4}, \frac{5}{4}\right)$ (центроид)
    
    Вычисляем векторы:
    \begin{align*}
    \overrightarrow{AP} &= \left(\frac{1}{4},\ \frac{1}{4},\ \frac{1}{4}\right) \\
    \overrightarrow{PB} &= \left(\frac{3}{4},\ \frac{3}{4},\ -\frac{5}{4}\right) \\
    \overrightarrow{PC} &= \left(-\frac{5}{4},\ \frac{3}{4},\ \frac{3}{4}\right) \\
    \overrightarrow{PD} &= \left(\frac{3}{4},\ -\frac{5}{4},\ \frac{3}{4}\right) \\
    \overrightarrow{AB} &= \left(1,\ 1,\ -1\right) \\
    \overrightarrow{AC} &= \left(-1,\ 1,\ 1\right) \\
    \overrightarrow{AD} &= \left(1,\ -1,\ 1\right)
    \end{align*}
    
    Вычисляем объёмы:
    \begin{align*}
    V_{PBCD} &= \frac{1}{6} \left| 
    \begin{vmatrix}
    \frac{3}{4} & \frac{3}{4} & -\frac{5}{4} \\
    -\frac{5}{4} & \frac{3}{4} & \frac{3}{4} \\
    \frac{3}{4} & -\frac{5}{4} & \frac{3}{4}
    \end{vmatrix} \right| = \frac{1}{6} > 0 \\
    V_{APCD} &= \frac{1}{6} \left| 
    \begin{vmatrix}
    \frac{1}{4} & \frac{1}{4} & \frac{1}{4} \\
    -1 & 1 & 1 \\
    1 & -1 & 1
    \end{vmatrix} \right| = \frac{1}{6} > 0  \\
    V_{ABPD} &= \frac{1}{6} \left| 
    \begin{vmatrix}
    1 & 1 & -1 \\
    \frac{1}{4} & \frac{1}{4} & \frac{1}{4} \\
    1 & -1 & 1
    \end{vmatrix} \right| = \frac{1}{6} > 0  \\
    V_{ABCP} &= \frac{1}{6} \left| 
    \begin{vmatrix}
    1 & 1 & -1 \\
    -1 & 1 & 1 \\
    \frac{1}{4} & \frac{1}{4} & \frac{1}{4}
    \end{vmatrix} \right| = \frac{1}{6} > 0 
    \end{align*}
    
    Проверяем: $V_{PBCD} + V_{APCD} + V_{ABPD} + V_{ABCP} = \frac{1}{6} + \frac{1}{6} + \frac{1}{6} + \frac{1}{6} = \frac{4}{6} = \frac{2}{3} = V_{ABCD}$
    

    \item \textbf{Точка на грани} \\
    Точка $P$ = $\alpha$$A$ + $\beta$$B$ + $\gamma$$C$ в грани $(ABC)$ \\
    $\alpha$ + $\beta$ + $\gamma$ = 1  \\  Тогда возьмём $\alpha$=0.4, $\beta$=0.3, $\gamma$=0.3 \\  $\alpha$ > 0,  $\beta$ > 0,  $\gamma$ > 0 \\
    $P(1, 1.6, 1)$ на грани $ABC$ (получена как $P = 0.4A + 0.3B + 0.3C$)
    
    Вычисляем векторы:
    \begin{align*}
    \overrightarrow{AP} &= (0,\ 0.6,\ 0) \\
    \overrightarrow{PB} &= (1,\ 0.4,\ -1) \\
    \overrightarrow{PC} &= (-1,\ 0.4,\ 1) \\
    \overrightarrow{PD} &= (1,\ -1.6,\ 1) \\
    \overrightarrow{AB} &= (1,\ 1,\ -1) \\
    \overrightarrow{AC} &= (-1,\ 1,\ 1) \\
    \overrightarrow{AD} &= (1,\ -1,\ 1)
    \end{align*}
    
    Вычисляем объёмы:
    \begin{align*}
    V_{PBCD} &= \frac{1}{6} \left| 
    \begin{vmatrix}
    1 & 0.4 & -1 \\
    -1 & 0.4 & 1 \\
    1 & -1.6 & 1
    \end{vmatrix} \right|= \frac{4}{15}> 0 \\
    V_{APCD} &= \frac{1}{6} \left| 
    \begin{vmatrix}
    0 & 0.6 & 0 \\
    -1 & 1 & 1 \\
    1 & -1 & 1
    \end{vmatrix} \right| = \frac{1}{5}> 0 \\
    V_{ABPD} &= \frac{1}{6} \left| 
    \begin{vmatrix}
    1 & 1 & -1 \\
    0 & 0.6 & 0 \\
    1 & -1 & 1
    \end{vmatrix} \right| = \frac{1}{5}> 0 \\
    V_{ABCP} &= \frac{1}{6} \left| 
    \begin{vmatrix}
    1 & 1 & -1 \\
    -1 & 1 & 1 \\
    0 & 0.6 & 0
    \end{vmatrix} \right| = 0 \quad \text{(точка в плоскости ABC)}
    \end{align*}

    Проверяем: $V_{PBCD} + V_{APCD} + V_{ABPD} + 0 = \frac{1}{5} + \frac{1}{5} + \frac{4}{15} = \frac{2}{3} = V_{ABCD}$

    \item \textbf{Точка на ребре} \\
    Возьмём точку $P(2,\ 0.2,\ 1.8)$ на ребре $BD$ (получена как $P = 0.9B + 0.1D$)
    
    Вычисляем векторы:
    \begin{align*}
    \overrightarrow{AP} &= (1,\ -0.8,\ 0.8) \\
    \overrightarrow{PB} &= (0,\ 1.8,\ -1.8) \\
    \overrightarrow{PC} &= (-2,\ 1.8,\ 0.2) \\
    \overrightarrow{PD} &= (0,\ -0.2,\ 0.2) \\
    \overrightarrow{AB} &= (1,\ 1,\ -1) \\
    \overrightarrow{AC} &= (-1,\ 1,\ 1) \\
    \overrightarrow{AD} &= (1,\ -1,\ 1)
    \end{align*}
    
    Вычисляем объёмы:
    \begin{align*}
    V_{PBCD} &= \frac{1}{6} \left| 
    \begin{vmatrix}
    0 & 1.8 & -1.8 \\
    -2 & 1.8 & 0.2 \\
    0 & -0.2 & 0.2
    \end{vmatrix} \right| = 0 \\
    V_{APCD} &= \frac{1}{6} \left| 
    \begin{vmatrix}
    1 & -0.8 & 0.8 \\
    -1 & 1 & 1 \\
    1 & -1 & 1
    \end{vmatrix} \right| = \frac{1}{15} > 0 \\
    V_{ABPD} &= \frac{1}{6} \left| 
    \begin{vmatrix}
    1 & 1 & -1 \\
    1 & -0.8 & 0.8 \\
    1 & -1 & 1
    \end{vmatrix} \right| = 0 > 0 \\
    V_{ABCP} &= \frac{1}{6} \left| 
    \begin{vmatrix}
    1 & 1 & -1 \\
    -1 & 1 & 1 \\
    1 & -0.8 & 0.8
    \end{vmatrix} \right| = \frac{3}{5}> 0
    \end{align*}

    \noindent
    $V_{ABCD} = V_{PBCD} + V_{APCD} + V_{ABPD} + V_{ABCP} = 0 + \frac{1}{15} + 0 + \frac{3}{5} = \frac{2}{3}$ \\
    Точка $P$ лежит в двух гранях $(ABD)$ и $(BCD)$, т.к. $P$ принадлежит $BD$. 
    В гранях $(ABD)$ и $(BCD)$ точки лежат в одной плоскости $\Rightarrow V_{PBCD} = 0$ и $V_{ABPD} = 0$ \\
    $V_{APCD} > 0$, $V_{ABCP} > 0$, $V_{APCD} + V_{ABCP} = \frac{2}{3}$ $\Rightarrow$ точка лежит на ребре $BD$
    
    
    \item \textbf{Точка совпадает с вершиной} \\
    Возьмём точку $P(1,\ 1,\ 1)$ (вершина $A$) \\ 
    Тогда $V_{PBCD} = V_{ABCD} = \frac{2}{3}$, $V_{APCD} = V_{ABPD} = V_{ABCP} = 0$ \\
    
    
    Вычисляем векторы:
    \begin{align*}
    \overrightarrow{AP} &= (0,\ 0,\ 0) \\
    \overrightarrow{PB} &= (1,\ 1,\ -1) \\
    \overrightarrow{PC} &= (-1,\ 1,\ 1) \\
    \overrightarrow{PD} &= (1,\ -1,\ 1) \\
    \overrightarrow{AB} &= (1,\ 1,\ -1) \\
    \overrightarrow{AC} &= (-1,\ 1,\ 1) \\
    \overrightarrow{AD} &= (1,\ -1,\ 1)
    \end{align*}
    
    Вычисляем объёмы:
    \begin{align*}
    V_{PBCD} &= \frac{1}{6} \left| 
    \begin{vmatrix}
    1 & 1 & -1 \\
    -1 & 1 & 1 \\
    1 & -1 & 1
    \end{vmatrix} \right| = \frac{2}{3} > 0 \\
    V_{APCD} &= \frac{1}{6} \left| 
    \begin{vmatrix}
    0 & 0 & 0 \\
    -1 & 1 & 1 \\
    1 & -1 & 1
    \end{vmatrix} \right| = 0 \\
    V_{ABPD} &= \frac{1}{6} \left| 
    \begin{vmatrix}
    1 & 1 & -1 \\
    0 & 0 & 0 \\
    1 & -1 & 1
    \end{vmatrix} \right| = 0 \\
    V_{ABCP} &= \frac{1}{6} \left| 
    \begin{vmatrix}
    1 & 1 & -1 \\
    -1 & 1 & 1 \\
    0 & 0 & 0
    \end{vmatrix} \right| = 0
    \end{align*}

    \noindent
    $V_{ABCD} = V_{PBCD} + V_{APCD} + V_{ABPD} + V_{ABCP} = \frac{2}{3} + 0 + 0 + 0 = \frac{2}{3}$ \\
    Точка $P$ лежит в трёх гранях $(ABC)$, $(ACD)$ и $(ABD)$, т.к. $P$ совпадает с $A$.

    
    \item \textbf{Точка снаружи тетраэдра} \\
    Возьмём точку $P(0,\ 2,\ 3)$
    
    Вычисляем векторы:
    \begin{align*}
    \overrightarrow{AP} &= (-1,\ 1,\ 2) \\
    \overrightarrow{PB} &= (2,\ 0,\ -3) \\
    \overrightarrow{PC} &= (0,\ 0,\ -1) \\
    \overrightarrow{PD} &= (2,\ -2,\ -1) \\
    \overrightarrow{AB} &= (1,\ 1,\ -1) \\
    \overrightarrow{AC} &= (-1,\ 1,\ 1) \\
    \overrightarrow{AD} &= (1,\ -1,\ 1)
    \end{align*}
    
    Вычисляем объёмы:
    \begin{align*}
    V_{PBCD} &= \frac{1}{6} \left| 
    \begin{vmatrix}
    2 & 0 & -3 \\
    0 & 0 & -1 \\
    2 & -2 & -1
    \end{vmatrix} \right| = \frac{2}{3} > 0 \\
    V_{APCD} &= \frac{1}{6} \left| 
    \begin{vmatrix}
    -1 & 1 & 2 \\
    -1 & 1 & 1 \\
    1 & -1 & 1
    \end{vmatrix} \right| = \frac{1}{3} > 0 \\
    V_{ABPD} &= \frac{1}{6} \left| 
    \begin{vmatrix}
    1 & 1 & -1 \\
    -1 & 1 & 2 \\
    1 & -1 & 1
    \end{vmatrix} \right| = 1 > 0 \\
    V_{ABCP} &= \frac{1}{6} \left| 
    \begin{vmatrix}
    1 & 1 & -1 \\
    -1 & 1 & 1 \\
    -1 & 1 & 2
    \end{vmatrix} \right| = \frac{1}{3} > 0
    \end{align*}
    
    Проверяем: $V_{PBCD} + V_{APCD} + V_{ABPD} + V_{ABCP} = \frac{2}{3} + \frac{1}{3} + 1 + \frac{1}{3} = \frac{2}{3} + \frac{1}{3} + \frac{3}{3} + \frac{1}{3} = \frac{7}{3} > \ V_{ABCD} = \frac{2}{3}$ \\
    Сумма объёмов больше объёма тетраэдра, значит точка находится снаружи. \\

\end{enumerate}

\subsection*{\textcolor{black}{Вывод}}
Метод объёмов работает корректно для всех пяти случаев (внутри, на грани, на ребре, в вершине, снаружи). Для его реализации необходимо вычислить 4 определителя размером 3×3. Возможна погрешность при работе с вещественными координатами в связи с ошибками округления.

\subsection*{\textcolor{black}{Код программ}}
\begin{lstlisting}[language=Python, caption={Функция вычисления объема тетраэдра на Python}]
def v(x):
    positive = x[0][0] * x[1][1] * x[2][2] + x[1][0] * x[0][2] * x[2][1] + x[0][1] * x[1][2] * x[2][0]
    negative = x[0][2] * x[1][1] * x[2][0] + x[0][1] * x[1][0] * x[2][2] + x[0][0] * x[1][2] * x[2][1]

    return (1 / 6) * abs(positive - negative)


line1, line2, line3 = list(map(float, input().split())), list(map(float, input().split())), list(
    map(float, input().split()))
matrix = [line1, line2, line3]

print(v(matrix))
\end{lstlisting}

\subsubsection*{\textcolor{blue}{Метод 3: Метод знаков смешанных произведений}}

\paragraph{Основная идея}
Составляем уравнение плоскости через три вершины каждой грани. Подставляем координаты точки $P$ и противоположной вершины и сравниваем знаки полученных значений. 

\paragraph{Описание математической реализации}
Для решения необоходимо найти уравнения всех четырех плоскостей, образующих грани тетраэдра ($ABC$, $ABD$, $ACD$, $BCD$). Затем для проверяемой точки $P$ подставить ее координаты в уравнения этих плоскостей: если точка $P$ лежит внутри тетраэдра, то она будет удовлетворять неравенствам, соответствующим всем четырем плоскостям (например, если вершина $A$ напротив плоскости $BCD$, то точка $P$ должна быть с той же стороны от $BCD$, что и $A$, и так для всех граней). Причем случай равенства 0 стоит рассматривать как принадлежность точки к плоскости. 

\begin{enumerate}[leftmargin=*, itemsep=0.5em]

	\item \textbf{Нахождение уравнений плоскостей} \\
	Найдем уравнения плоскостей с 3 известными точками через определитель матрицы вида:
	\begin{align*}
    \left|\begin{matrix}
    x-x_{1}&y-y_{1}&z-z_{1}\\ 
    x_{2}-x_{1}&y_{2}-y_{1}&z_{2}-z_{1}\\ 
    x_{3}-x_{1}&y_{3}-y_{1}&z_{3}-z_{1}
    \end{matrix}\right|
    \end{align*}
	Для плоскости $ABC$:
	\begin{align*}
    \left|
    \begin{matrix}
    x-1 & y-1 & z-1\\
    1 & 1 & -1\\
    -1 & 1 & 1
    \end{matrix}
    \right|=$x + z - 2$
    \end{align*}
	Для плоскости $ACD$:
    \begin{align*}
    \left|
    \begin{matrix}
    x-1 & y-1 & z-1\\
    -1 & 1 & 1\\
    1 & -1 & 1
    \end{matrix}
    \right|=$x + y - 2$
    \end{align*}
	Для плоскости $ABD$:
    \begin{align*}
	\left|
    \begin{matrix}
    x-1 & y-1 & z-1\\
    -1 & 1 & 1\\
    1 & -1 & 1
    \end{matrix}
	\right|=$- y - z + 2$
    \end{align*}
	Для плоскости $BCD$:
    \begin{align*}
	\left|
    \begin{matrix}
    x-2 & y-2 & z\\
    -2 & 0 & 2\\
    0 & -2 & 2
    \end{matrix}
	\right|=$x + y + z - 4$
	\end{align*}

    Также сразу найдем знак для вершины, противопложной каждой из плоскосетй: \\
    \begin{itemize}
        \item $ABC$ - $D(2, 0, 2)$: $2 + 2 - 2 > 0$ \\
        \item $ACD$ - $B(2, 2, 0)$: $2 + 2 - 2 > 0$ \\
        \item $ABD$ - $C(0, 2, 2)$: $- 2 - 2 + 2 < 0$ \\
        \item $BCD$ - $A(1, 1, 1)$: $1 + 1 + 1 - 4 < 0$ \\
    \end{itemize}


    \item \textbf{Точка внутри тетраэдра} \\
     Возьмём точку $P\left(\frac{5}{4}, \frac{5}{4}, \frac{5}{4}\right)$ (центроид). Проверим для каждой плоскости \\
     \begin{itemize}
        \item $ABC$: $\frac{5}{4} + \frac{5}{4} - 2 > 0$ - знак совпал \\
        \item $ACD$: $\frac{5}{4} + \frac{5}{4} - 2 > 0$ - знак совпал \\
        \item $ABD$: $- \frac{5}{4} - \frac{5}{4} + 2 < 0$ - знак совпал \\
        \item $BCD$: $\frac{5}{4} + \frac{5}{4} + \frac{5}{4} - 4 < 0$ - знак совпал \\
    \end{itemize}
    Для всех плоскостей знак совпал, значит точка лежит внутри тетраэдра

    \item \textbf{Точка внутри тетраэдра} \\
     Возьмём точку $P(1, 1.6, 1)$. Проверим для каждой плоскости \\
     \begin{itemize}
        \item $ABC$: $1 + 1 - 2 = 0$ - точка лежит на $ABC$  \\
        \item $ACD$: $1 + 1.6 - 2 > 0$ - знак совпал \\
        \item $ABD$: $- 1.6 - 1 + 2 < 0$ - знак совпал \\
        \item $BCD$: $1 + 1.6 + 1 - 4 < 0$ - знак совпал  \\
    \end{itemize}
    Для всех плоскостей знак совпал, значит точка лежит внутри тетраэдра

    \item \textbf{Точка на ребре} \\
     Возьмём точку $P(2, 0.2, 1.8)$ на ребре $BD$. Проверим для каждой плоскости \\
     \begin{itemize}
        \item $ABC$: $2 + 1.8 - 2 > 0$ -  знак совпал  \\
        \item $ACD$: $2 + 0.2 - 2 > 0$ - знак совпал \\
        \item $ABD$: $- 0.2 - 1.8 + 2 = 0$ - точка лежит на $ABD$ \\
        \item $BCD$: $2 + 0.2 + 1.8 - 4 = 0$ - точка лежит на $BCD$  \\
    \end{itemize}
    Для всех плоскостей знак совпал, значит точка лежит внутри тетраэдра
     
     \item \textbf{Точка совпадает с вершиной} \\
     Возьмём точку $P(1, 1, 1)$ (вершина). Проверим для каждой плоскости \\
     \begin{itemize}
        \item $ABC$: $1 + 1 - 2 = 0$ -  точка лежит на $ABC$   \\
        \item $ACD$: $1 + 1 - 2 = 0$ - точка лежит на $ACD$ \\
        \item $ABD$: $- 1 - 1 + 2 = 0$ - точка лежит на $ABD$ \\
        \item $BCD$: $1 + 1 + 1 - 4 < 0$ - знак совпал  \\
    \end{itemize}
    Для всех плоскостей знак совпал, значит точка лежит внутри тетраэдра

     \item \textbf{Точка снаружи тетраэдра} \\
     Возьмём точку $P(0, 2, 3)$ (вершина). Проверим для каждой плоскости \\
     \begin{itemize}
        \item $ABC$: $0 + 3 - 2 > 0$ -  знак совпал   \\
        \item $ACD$: $0 + 2 - 2 = 0$ - точка лежит на $ACD$ \\
        \item $ABD$: $- 0 - 3 + 2 < 0$ - знак совпал \\
        \item $BCD$: $0 + 2 + 3 - 4 > 0$ - знак не совпал  \\
    \end{itemize}
    Знак не совпал для $BCD$, значит точка снаружи тетраэдра

\end{enumerate}

\subsection*{\textcolor{black}{Вывод}}
Метод уравнений плоскостей работает корректно для всех пяти случаев (внутри, на грани, на ребре, в вершине, снаружи). Для его реализации необходимо вычислить уравнения 4 плоскостей тетраэдра и сравнить знаки.

\subsection*{\textcolor{black}{Код программ}}
\begin{lstlisting}[language=Java, caption={Решение методом уравнений плоскостей на Java}]
// Main.java

import plainEquations.Plain;
import plainEquations.Point;
import java.lang.Math;

public class Main {
    public static void main(String[] args) {
        Point[] testPoints = {
            new Point(1.25, 1.25, 1.25), 
            new Point(1., 1.6, 1.), 
            new Point(2., 0.2, 1.8), 
            new Point(1., 1., 1.), 
            new Point(0., 2., 3.)
        };
     
        check(testPoints);
    }


    public static void check(Point[] testPoints) {
        Point A = new Point(1., 1., 1.);
        Point B = new Point(2., 2., 0.);
        Point C = new Point(0., 2., 2.);
        Point D = new Point(2., 0., 2.);
        
        Plain ABC = new Plain(D, point -> Math.signum(point.x() + point.z() - 2)); 
        Plain ACD = new Plain(B, point -> Math.signum(point.x() + point.y() - 2)); 
        Plain ABD = new Plain(C, point -> Math.signum(- point.y() - point.z() + 2)); 
        Plain BCD = new Plain(A, point -> Math.signum(point.x() + point.y() + point.z() - 4)); 
        
        for (Point point: testPoints) {
            System.out.println(point);
            System.out.println("ABC: " + ABC.check(point));
            System.out.println("ACD: " + ACD.check(point));
            System.out.println("ABD: " + ABD.check(point));
            System.out.println("BCD: " + BCD.check(point));
            System.out.println((ABC.check(point) && ACD.check(point) && ABD.check(point) && BCD.check(point) ? "точка внутри" : "точка снаружи"));
            System.out.println("---");
        }
    }
}

// Plain.java

package plainEquations;

import java.util.function.Function;

public class Plain {
    Point oppositeVertex;
    Function<Point, Double> getSign;

    public Plain(Point oppositeVertex, Function<Point, Double> getSign) {
        this.oppositeVertex = oppositeVertex;
        this.getSign = getSign;
    }

    public boolean check(Point checkPoint) {
        Double sign1 = this.getSign.apply(this.oppositeVertex);
        Double sign2 = this.getSign.apply(checkPoint);
        return sign1.equals(sign2) || sign2 == 0;
    }
} 

// Point.java

package plainEquations;

public record Point(double x, double y, double z){};
\end{lstlisting}

\subsubsection*{\textcolor{blue}{Метод 4: Метод барицентрических координат}}

\paragraph{Основная идея}
Точка представляется с помощью линейной комбинации вершин многогранника, используя веса для каждой вершины.
Эти коэффициенты называются барицентрическими координатами.

\paragraph{Описание математической реализации}
Для нахождения барицентрических координат и проверки положения точки $P$ относительно тетраэдра составляется система уравнений.Полученные значения коэффициентов позволяют определить положение точки.

\subsection*{\textcolor{black}{1. Точка внутри тетраэдра}}

Возьмем точку $P(1.25, 1.25, 1.25)$, которая лежит внутри тетраэдра.

Составим систему уравнений для барицентрических координат:

\[
\left\{
\begin{aligned}
\lambda_1 + \lambda_2 + \lambda_3 + \lambda_4 &= 1, \\
\lambda_1 \cdot 1 + \lambda_2 \cdot 2 + \lambda_3 \cdot 0 + \lambda_4 \cdot 2 &= 1.25, \\
\lambda_1 \cdot 1 + \lambda_2 \cdot 2 + \lambda_3 \cdot 2 + \lambda_4 \cdot 0 &= 1.25, \\
\lambda_1 \cdot 1 + \lambda_2 \cdot 0 + \lambda_3 \cdot 2 + \lambda_4 \cdot 2 &= 1.25.
\end{aligned}
\right.
\]

Приведем систему к удобному виду:

1. Во втором уравнении:
\[
\lambda_1 + 2\lambda_2 + 2\lambda_4 = 1.25
\]

2. В третьем уравнении:
\[
\lambda_1 + 2\lambda_2 + 2\lambda_3 = 1.25
\]

3. В четвертом уравнении:
\[
\lambda_1 + 2\lambda_3 + 2\lambda_4 = 1.25
\]

Теперь выразим \( \lambda_1 \) из первого уравнения:

\[
\lambda_1 = 1 - \lambda_2 - \lambda_3 - \lambda_4
\]

Подставим это выражение для \( \lambda_1 \) в оставшиеся уравнения.

1. Во втором уравнении:
\[
(1 - \lambda_2 - \lambda_3 - \lambda_4) + 2\lambda_2 + 2\lambda_4 = 1.25
\]
Упрощаем:
\[
1 - \lambda_2 - \lambda_3 - \lambda_4 + 2\lambda_2 + 2\lambda_4 = 1.25
\]
\[
\lambda_2 + \lambda_4 - \lambda_3 = 0.25
\]

2. В третьем уравнении:
\[
(1 - \lambda_2 - \lambda_3 - \lambda_4) + 2\lambda_2 + 2\lambda_3 = 1.25
\]
Упрощаем:
\[
1 - \lambda_2 - \lambda_3 - \lambda_4 + 2\lambda_2 + 2\lambda_3 = 1.25
\]
\[
\lambda_2 + \lambda_3 - \lambda_4 = 0.25
\]

3. В четвертом уравнении:
\[
(1 - \lambda_2 - \lambda_3 - \lambda_4) + 2\lambda_3 + 2\lambda_4 = 1.25
\]
Упрощаем:
\[
1 - \lambda_2 - \lambda_3 - \lambda_4 + 2\lambda_3 + 2\lambda_4 = 1.25
\]
\[
\lambda_3 + \lambda_4 - \lambda_2 = 0.25
\]

Теперь у нас есть система из трех уравнений:

\[
\left\{
\begin{aligned}
\lambda_2 + \lambda_4 - \lambda_3 &= 0.25, \\
\lambda_2 + \lambda_3 - \lambda_4 &= 0.25, \\
\lambda_3 + \lambda_4 - \lambda_2 &= 0.25.
\end{aligned}
\right.
\]

Сложим все три уравнения:

\[
(\lambda_2 + \lambda_4 - \lambda_3) + (\lambda_2 + \lambda_3 - \lambda_4) + (\lambda_3 + \lambda_4 - \lambda_2) = 0.25 + 0.25 + 0.25
\]
\[
2\lambda_3 + 2\lambda_4 = 0.75
\]
\[
\lambda_3 + \lambda_4 = 0.375
\]

Теперь подставим \( \lambda_3 + \lambda_4 = 0.375 \) в два других уравнения:

- Подставим в \( \lambda_2 + \lambda_4 - \lambda_3 = 0.25 \):
\[
\lambda_2 + \lambda_4 - \lambda_3 = 0.25 \quad \Rightarrow \quad \lambda_2 + (0.375 - \lambda_3) - \lambda_3 = 0.25
\]
\[
\lambda_2 = 2\lambda_3 - 0.125
\]

- Подставим в \( \lambda_2 + \lambda_3 - \lambda_4 = 0.25 \):
\[
\lambda_2 + \lambda_3 - (0.375 - \lambda_3) = 0.25
\]
\[
\lambda_2 + 2\lambda_3 - 0.375 = 0.25
\]
\[
\lambda_2 = 0.625 - 2\lambda_3
\]

Теперь подставим \( \lambda_2 = 0.625 - 2\lambda_3 \) в \( \lambda_2 = 2\lambda_3 - 0.125 \):

\[
0.625 - 2\lambda_3 = 2\lambda_3 - 0.125
\]
\[
0.75 = 4\lambda_3
\]
\[
\lambda_3 = 0.1875
\]

Теперь подставим \( \lambda_3 = 0.1875 \) в \( \lambda_2 = 0.625 - 2\lambda_3 \):

\[
\lambda_2 = 0.625 - 2 \cdot 0.1875 = 0.25
\]

Подставим \( \lambda_3 = 0.1875 \) в \( \lambda_3 + \lambda_4 = 0.375 \):

\[
0.1875 + \lambda_4 = 0.375 \quad \Rightarrow \quad \lambda_4 = 0.1875
\]

Теперь находим \( \lambda_1 \):

\[
\lambda_1 = 1 - \lambda_2 - \lambda_3 - \lambda_4 = 1 - 0.25 - 0.1875 - 0.1875 = 0.375
\]

### Итог:

Таким образом, для точки \( P(1.25, 1.25, 1.25) \) барицентрические координаты:

\[
\lambda_1 = 0.375, \quad \lambda_2 = 0.25, \quad \lambda_3 = 0.1875, \quad \lambda_4 = 0.1875
\]

Это означает, что точка \( P \) лежит внутри тетраэдра.



\subsection*{\textcolor{black}{2. Точка на грани тетраэдра}}

Возьмем точку $P(1, 1.6, 1)$, которая лежит на грани тетраэдра.

Составим и решим систему уравнений для барицентрических координат:

\[
\left\{
\begin{aligned}
\lambda_1 + \lambda_2 + \lambda_3 + \lambda_4 &= 1, \\
\lambda_1 + 2\lambda_2 + 2\lambda_4 &= 1, \\
\lambda_1 + 2\lambda_2 + 2\lambda_3 &= 1.6, \\
\lambda_1 + 2\lambda_3 + 2\lambda_4 &= 1.
\end{aligned}
\right.
\]

Подставим \( \lambda_1 = 1 - \lambda_2 - \lambda_3 - \lambda_4 \) в оставшиеся уравнения.

Упрощая систему, получаем:

\[
\left\{
\begin{aligned}
\lambda_2 + \lambda_4 - \lambda_3 &= 0, \\
\lambda_2 + \lambda_3 - \lambda_4 &= 0.6, \\
\lambda_3 + \lambda_4 - \lambda_2 &= 0.
\end{aligned}
\right.
\]

Сложив все уравнения, находим \( \lambda_3 = 0.3 \). Подставляем \( \lambda_3 = 0.3 \) в два других уравнения:

\[
\lambda_2 + \lambda_4 = 0.3 \quad \text{и} \quad \lambda_2 - \lambda_4 = 0.3.
\]

Сложив эти уравнения, получаем \( \lambda_2 = 0.3 \). Подставляем в \( \lambda_2 + \lambda_4 = 0.3 \), получаем \( \lambda_4 = 0 \).

Теперь, зная \( \lambda_2 = 0.3 \), \( \lambda_3 = 0.3 \), \( \lambda_4 = 0 \), находим \( \lambda_1 \):

\[
\lambda_1 = 1 - \lambda_2 - \lambda_3 - \lambda_4 = 1 - 0.3 - 0.3 - 0 = 0.4.
\]

\textbf{Итог}: Для точки $P(1, 1.6, 1)$ барицентрические координаты:

\[
\lambda_1 = 0.4, \quad \lambda_2 = 0.3, \quad \lambda_3 = 0.3, \quad \lambda_4 = 0.
\]

Это означает, что точка \( P \) лежит на грани тетраэдра, тк один из ее коэффициентов равен нулю.


\subsection*{\textcolor{black}{3. Точка на ребре тетраэдра}}

Рассмотрим точку
\[
P(2,\,0.2,\,1.8).
\]


Найдём её барицентрические координаты $\lambda_1,\lambda_2,\lambda_3,\lambda_4$ относительно вершин тетраэдра.


Составим систему уравнений для барицентрических координат:
\[
\begin{cases}
\lambda_1 + \lambda_2 + \lambda_3 + \lambda_4 = 1, \\
\lambda_1 \cdot 1 + \lambda_2 \cdot 2 + \lambda_3 \cdot 0 + \lambda_4 \cdot 2 = 2, \\
\lambda_1 \cdot 1 + \lambda_2 \cdot 2 + \lambda_3 \cdot 2 + \lambda_4 \cdot 0 = 0.2, \\
\lambda_1 \cdot 1 + \lambda_2 \cdot 0 + \lambda_3 \cdot 2 + \lambda_4 \cdot 2 = 1.8.
\end{cases}
\]


Из первого уравнения выразим $\lambda_1$:
\[
\lambda_1 = 1 - \lambda_2 - \lambda_3 - \lambda_4.
\]


Подставим это выражение в остальные три уравнения системы.


После упрощения получаем эквивалентную систему:
\[
\begin{cases}
\lambda_2 - \lambda_3 + \lambda_4 = 1, \\
\lambda_2 + \lambda_3 - \lambda_4 = -0.8, \\
-\lambda_2 + \lambda_3 + \lambda_4 = 0.8.
\end{cases}
\]


Складывая первые два уравнения, находим:
\[
2\lambda_2 = 0.2 \quad \Rightarrow \quad \lambda_2 = 0.1.
\]


Подставляя это значение в первые и третье уравнения, получаем систему:
\[
\begin{cases}
\lambda_4 - \lambda_3 = 0.9, \\
\lambda_3 + \lambda_4 = 0.9.
\end{cases}
\]


Отсюда следует:
\[
\lambda_3 = 0, \qquad \lambda_4 = 0.9.
\]


Найдём $\lambda_1$:
\[
\lambda_1 = 1 - 0.1 - 0 - 0.9 = 0.
\]


\textbf{Итог}: Для точки $P(2,0.2,1.8)$ барицентрические координаты:

\[
\lambda_1 = 0, \quad \lambda_2 = 0.1, \quad \lambda_3 = 0, \quad \lambda_4 = 0.9.
\]

Это означает, что точка \( P \) лежит на ребре тетраэдра, тк два ее коэффициента равны нулю.

\subsection*{\textcolor{black}{4. Точка в вершине тетраэдра}}
Рассмотрим точку
\[
P(1,\,1,\,1).
\]


Найдём её барицентрические координаты $\lambda_1,\lambda_2,\lambda_3,\lambda_4$ относительно вершин тетраэдра.

\[
\begin{cases}
\lambda_1 + \lambda_2 + \lambda_3 + \lambda_4 = 1, \\
\lambda_1 + 2\lambda_2 + 2\lambda_4 = 1, \\
\lambda_1 + 2\lambda_2 + 2\lambda_3 = 1, \\
\lambda_1 + 2\lambda_3 + 2\lambda_4 = 1.
\end{cases}
\]


Вычтем второе уравнение из третьего:
\[
2\lambda_3 - 2\lambda_4 = 0 \quad \Rightarrow \quad \lambda_3 = \lambda_4.
\]


Вычтем второе уравнение из четвёртого:
\[
2\lambda_3 - 2\lambda_2 = 0 \quad \Rightarrow \quad \lambda_2 = \lambda_3.
\]


Следовательно,
\[
\lambda_2 = \lambda_3 = \lambda_4.
\]


Подставим это в первое уравнение:
\[
\lambda_1 + 3\lambda_2 = 1. \tag{1}
\]


Подставим во второе уравнение:
\[
\lambda_1 + 4\lambda_2 = 1. \tag{2}
\]


Вычитая (1) из (2), получаем:
\[
\lambda_2 = 0.
\]


Тогда
\[
\lambda_1 = 1.
\]


\textbf{Итог}: Для точки $P(1, 1, 1)$ барицентрические координаты:

\[
\lambda_1 = 1, \quad \lambda_2 = 0, \quad \lambda_3 = 0, \quad \lambda_4 = 0.
\]
Это значит, что точка находится в вершине тетраэдра, тк три его координаты равны нулю


\subsection*{\textcolor{black}{5. Точка вне тетраэдра}}
\[
P(0,\,2,\,3).
\]

Найдём её барицентрические координаты $\lambda_1, \lambda_2, \lambda_3, \lambda_4$, решив систему уравнений:
\[
\begin{cases}
\lambda_1 + \lambda_2 + \lambda_3 + \lambda_4 = 1, \\
\lambda_1 + 2\lambda_2 + 2\lambda_4 = 0, \\
\lambda_1 + 2\lambda_2 + 2\lambda_3 = 2, \\
\lambda_1 + 2\lambda_3 + 2\lambda_4 = 3.
\end{cases}
\]

Из первого уравнения выразим $\lambda_1$:
\[
\lambda_1 = 1 - \lambda_2 - \lambda_3 - \lambda_4.
\]

Подставим это выражение в остальные три уравнения.

После упрощения получаем систему:
\[
\begin{cases}
\lambda_2 - \lambda_3 + \lambda_4 = -1, \\
\lambda_2 + \lambda_3 - \lambda_4 = 0, \\
-\lambda_2 + \lambda_3 + \lambda_4 = 1.
\end{cases}
\]

Сложим второе и третье уравнения:
\[
2\lambda_3 = 1 \quad \Rightarrow \quad \lambda_3 = 0.5.
\]

Подставим $\lambda_3 = 0.5$ во второе уравнение:
\[
\lambda_2 + 0.5 - \lambda_4 = 0 \quad \Rightarrow \quad \lambda_2 = \lambda_4 - 0.5.
\]

Подставим это выражение в первое уравнение системы:
\[
(\lambda_4 - 0.5) - 0.5 + \lambda_4 = -1,
\]
откуда
\[
2\lambda_4 = 0 \quad \Rightarrow \quad \lambda_4 = 0.
\]

Тогда
\[
\lambda_2 = -0.5.
\]

Найдём $\lambda_1$:
\[
\lambda_1 = 1 - (-0.5) - 0.5 - 0 = 1.
\]

\textbf{Итог}: Для точки $P(0, 2, 3)$ барицентрические координаты:

\[
\lambda_1 = 1, \quad \lambda_2 = -0.5, \quad \lambda_3 = 0.5, \quad \lambda_4 = 0.
\]

Это означает, что точка \( P \) лежит вне тетраэдра, тк один из ее коэффициентов отрицателен.

\section*{Вывод}

Метод барицентрических координат позволяет корректно определить положение точки относительно тетраэдра для всех пяти случаев: внутри, на грани, на ребре, в вершине и снаружи. Для реализации необходимо вычислить барицентрические координаты точки относительно вершин тетраэдра и проанализировать их знаки.

\setcounter{lstlisting}{3}
\subsection*{\textcolor{black}{Код программ}}

\begin{lstlisting}[caption={Решение методом барицентрических координат на Python}]
import numpy as np

def barycentric_tetrahedron(P, A, B, C, D):
    M = np.array([A, B, C, D]).T
    M = np.vstack((M, np.ones(4)))
    P_ext = np.array([*P, 1])

    coords = np.linalg.solve(M, P_ext)
    return coords


def point_position_tetrahedron(P, A, B, C, D, tol=1e-8):
    u, v, w, t = barycentric_tetrahedron(P, A, B, C, D)
    bary = np.array([u, v, w, t])

    if np.all(bary > tol):
        return "Inside "
    zeros = np.isclose(bary, 0, atol=tol)
    num_zeros = np.sum(zeros)

    if num_zeros == 1:
        return "On face"
    elif num_zeros == 2:
        return "On edge"
    elif num_zeros == 3:
        return "On vertex"
    else:
        return "Outside"


A = np.array([1, 1, 1])
B = np.array([2, 2, 0])
C = np.array([0, 2, 2])
D = np.array([2, 0, 2])

P_inside  = np.array([1.25, 1.25, 1.25])
P_face    = np.array([1, 1.6, 1])
P_edge    = np.array([1, 1, 1])
P_vertex  = A
P_outside = np.array([0, 2, 3])

print(point_position_tetrahedron(P_inside,  A, B, C, D))
print(point_position_tetrahedron(P_face,    A, B, C, D))
print(point_position_tetrahedron(P_vertex,  A, B, C, D))
print(point_position_tetrahedron(P_outside, A, B, C, D))
\end{lstlisting}

\subsection*{\textcolor{blue}{Тестовые точки}}
\noindent
\setlength{\tabcolsep}{14pt}
\begin{tabular}{|c|c|}
\hline
\textbf{Тестовая точка} & \textbf{Координаты} \\
\hline
Внутри & $\left(\frac{5}{4}, \frac{5}{4}, \frac{5}{4}\right)$ \\
\hline
На грани & $(1,1.6,1)$ \\
\hline
На ребре & $(2,0.2,1.8)$ \\
\hline
Совпадает с вершиной & $(1,1,1)$ \\
\hline
Снаружи & $(0,2,3)$ \\
\hline
\end{tabular}



\subsection*{\textcolor{blue}{Сравнение методов}}
\noindent
\setlength{\tabcolsep}{14pt}
\begin{tabular}{|c|c|}
\hline
\textbf{Метод} & \textbf{Плюсы с минусы} \\
\hline
Метод объёмов (через смешанное произведение) & $\left(\frac{5}{4}, \frac{5}{4}, \frac{5}{4}\right)$ \\
\hline
Метод знаков смешанных произведений &  \\
\hline
Метод уравнений плоскостей &  \\
\hline
Альтернативные подходы &  \\
\hline

\end{tabular}

\subsection*{\textcolor{blue}{Формулировка прикладной задачи}}
В компьютерной графике задача определения принадлежности точки тетраэдру применяется для проверки видимости объектов. При рендеринге сцены нужно узнать, попадает ли объект (или его вершина) внутрь пирамиды видимости камеры. Эта пирамида разбивается на несколько тетраэдров, и для каждого из них проверяется, находится ли точка внутри, используя уравнения плоскостей граней. Если точка лежит внутри хотя бы одного тетраэдра, объект считается видимым и обрисовывается. Такой способ помогает быстро отсекать невидимые части сцены и повышает производительность рендеринга.
\begin{figure}[H] 
    \centering
    \includegraphics[width=\textwidth]{camera.jpg}
    \caption{Пирамида видимости камеры (для прикладной задачи)}
    \label{fig:camera}
\end{figure}

\subsection*{\textcolor{blue}{Визуализация результатов}}
Для визуализации был использован сервис Geogebra
\begin{figure}[H] 
    \centering
    \includegraphics[width=\textwidth]{tetr-1.png}
    \caption{Тетраэдр ракурс 1}
    \label{fig:tetr-1}
\end{figure}

\begin{figure}[H] 
    \centering
    \includegraphics[width=\textwidth]{tetr-2.png}
    \caption{Тетраэдр ракурс 2}
    \label{fig:tetr-2}
\end{figure}

\end{document}