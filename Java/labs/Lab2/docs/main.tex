%>>>>>>>>>>>>>>>>>>>>>>>>>> ПЕРЕМЕННЫЕ >>>>>>>>>>>>>>>>>>>>>>>>>>>>>>>>>>>
%>>>>> Информация о кафедре
%\newcommand{\year}{2021 г.}  % Год устанавливается автоматически
\newcommand{\city}{Санкт-Петербург}  %  Футер, нижний колонтитул на титульном листе
\newcommand{\university}{Национальный исследовательский университет ИТМО}  % первая строка
\newcommand{\department}{Факультет программной инженерии и компьютерной техники}  % Вторая строка
\newcommand{\major}{Направление программная инженерия}  % Треьтя строка
% Пусть будет. Проще закоментить лишнее.
\newcommand{\education}{Образовательная программа системное и прикладное программное обеспечение}  % четвертая строка
%\newcommand{\specialization}{}  % пятая строка

%<<<<< Информация о кафедре

%>>>>> Назание работы
\newcommand{\reporttype}{ОТЧЕТ ПО ЛАБОРАТОРНОЙ РАБОТЕ} % тип работы, (главный заголовок титульного листа)
\newcommand{\lab}{Лабораторная работа}          % вид работы
\newcommand{\labnumber}{№ 2}                    % порядковый номер работы
\newcommand{\subject}{Программирование}         % учебный предмет
\newcommand{\labtheme}{Принципы ООП}            % Тема лабораторной работы
\newcommand{\variant}{№ 3109993}                % номер варианта работы

\newcommand{\student}{Бых Даниил Максимович}    % определение ФИО студента
\newcommand{\studygroup}{P3109}                 % определение учебной группы
\newcommand{\teacher}{% принимающий
    Гаврилов А. В.,\\[1mm]% ФИО лектора
    Наумова Н. А.% ФИО практика
}
%<<<<<<<<<<<<<<<<<<<<<<<<<< ПЕРЕМЕННЫЕ <<<<<<<<<<<<<<<<<<<<<<<<<<<<<<<<<<<


%>>>>>>>>>>>>>>>>>>>>>> ПРЕАМБУЛА >>>>>>>>>>>>>>>>>>>>>>>>>
\documentclass[14pt,final,oneside]{extreport}% класс документа, характеристики
%>>>>> Разметка документа
\usepackage[a4paper, mag=1000, left=3cm, right=1.5cm, top=2cm, bottom=2cm, headsep=0.7cm, footskip=1cm]{geometry} % По ГОСТу: left>=3cm, right=1cm, top=2cm, bottom=2cm,
\linespread{1} % межстройчный интервал по ГОСТу := 1.5
%<<<<< Разметка документа

\setlength{\parindent}{1.25cm}

%>>>>> babel c языковым пакетом НЕ должны быть первым импортируемым пакетом
\usepackage[utf8]{inputenc}
\usepackage[T1,T2A]{fontenc}
\usepackage[russian]{babel}
% \usepackage{lmodern}
%<<<<<

%\usepackage{cmap} %поиск в pdf

%>>>...>> прочие полезные пакеты
\usepackage{amsmath,amsthm,amssymb}
\usepackage{mathtext}
\usepackage{braket}
\usepackage{indentfirst}
\usepackage{graphicx}
\usepackage{float}
\usepackage{changepage}
\graphicspath{{assets}}
\DeclareGraphicsExtensions{.pdf,.png,.jpg}
%\usepackage{bookmark}

\usepackage[dvipsnames]{xcolor}
\usepackage{hyperref}  % Использование ссылок
\hypersetup{%  % Настройка разметки ссылок
    colorlinks=true,
    linkcolor=blue,
    filecolor=magenta,
    urlcolor=magenta,
%pdftitle={Overleaf Example},
%pdfpagemode=FullScreen,
}

\usepackage{diagbox}
\usepackage[letterspace=150]{microtype} % Спэйсинг (межбуквенный интервал для саголовка) \lsstyle
% \usepackage{csvsimple} %импорт содержимого таблицы из csv

%>>> верстка в 2 колонки
\usepackage{multicol} % многоколоночная верстка
\setlength{\columnsep}{.15\textwidth} % определение ширины разделителя между колонками

\usepackage{tikz} % пакет для векторной графики, чтобы рисовать красивый разделитель колонок
% %> кастомный разделитель колонок
% \usetikzlibrary{arrows.meta,decorations.pathmorphing,backgrounds,positioning,fit,petri}
% \usepackage{multicolrule} % Для кастомизации разделителя колонок
% \SetMCRule{                     % кастомизация разделителя колонок multicolrule
%     width=2pt,
%     custom-line={               % Tikz код для кастомизации линии разделителя
%         \draw [                 % Рисовать
%             decorate,           % декорированную (требуются спец настройки пакетов tikz (см. импорт выше)
%             decoration={        % вид декорирования
%                 snake, % Тип - змейка (волнистая)
%                 amplitude=.5mm, % ширина волн
%                 pre length=0mm, % участок прямой линии от начала
%                 %segment length=0mm, % учасок волнистой линии
%                 post length=0mm % участок прямой линии от конца
%             },
%             line width=1pt,
%             step=10pt
%         ] 
%         (TOP) to (BOT); % сверху и до низа колонки
%     }, 
%     extend-top=-5pt, % Вылезти за верхнюю границу колонки 
%     extend-bot=-7pt % Вылезти за нижнюю границу колонки  
% }
%% < кастомный разделитель колонок
%%<<< верстка в 2 колонки

%>>>>> Использование листингов
\usepackage{listings}
\usepackage{caption}
\DeclareCaptionFont{white}{\color{white}}
\DeclareCaptionFormat{listing}{\colorbox{gray}{\parbox{\textwidth}{#1#2#3}}}

\captionsetup[lstlisting]{format=listing,labelfont=white,textfont=white} % Настройка вида описаний
\lstset{  % Настройки вида листинга
    inputencoding=utf8, extendedchars=\true, keepspaces = true, % поддержка кириллицы и пробелов в комментариях
    language={},            % выбор языка для подсветки (здесь это Pascal)
    basicstyle=\small\sffamily, % размер и начертание шрифта для подсветки кода
    numbers=left,               % где поставить нумерацию строк (слева\справа)
    numberstyle=\tiny,          % размер шрифта для номеров строк
    stepnumber=1,               % размер шага между двумя номерами строк
    numbersep=5pt,              % как далеко отстоят номера строк от подсвечиваемого кода
    backgroundcolor=\color{white}, % цвет фона подсветки - используем \usepackage{color}
    showspaces=false,           % показывать или нет пробелы специальными отступами
    showstringspaces=false,     % показывать илигнет пробелы в строках
    showtabs=false,             % показывать или нет табуляцию в строках
    frame=single,               % рисовать рамку вокруг кода
    tabsize=2,                  % размер табуляции по умолчанию равен 2 пробелам
    captionpos=t,               % позиция заголовка вверху [t] или внизу [b]
    breaklines=true,            % автоматически переносить строки (да\нет)
    breakatwhitespace=false,    % переносить строки только если есть пробел
    escapeinside={\%*}{*)}      % если нужно добавить комментарии в коде
}

\definecolor{codegreen}{rgb}{0,0.6,0}
\definecolor{codegray}{rgb}{0.5,0.5,0.5}
\definecolor{codepurple}{rgb}{0.58,0,0.82}
\definecolor{backcolour}{rgb}{0.95,0.95,0.92}

\lstdefinestyle{mystyle}{
backgroundcolor=\color{backcolour},
commentstyle=\color{codegreen},
keywordstyle=\color{magenta},
numberstyle=\tiny\color{codegray},
stringstyle=\color{codepurple},
basicstyle=\ttfamily\footnotesize,
breakatwhitespace=false,
breaklines=true,
captionpos=b,
keepspaces=true,
numbers=left,
numbersep=5pt,
showspaces=false,
showstringspaces=false,
showtabs=false,
tabsize=2
}
\lstset{style=mystyle}
%<<<<< Использование листингов


\sloppy % Решение проблем с переносами (с. 119 книга Львовского)
\emergencystretch=25pt


%>>>>>>>>>>>>>>>> ДОПОЛНИТЕЛЬНЫЕ КОМАНДЫ {Для соответствия ГОСТ} >>>>>>>>>>>>>>
%>>>>>> математические функции для удобства
\newcommand{\tx}{\text}
\newcommand{\eps}{\varepsilon}
\renewcommand{\phi}{\varphi}
\newcommand{\limit}{\displaystyle\lim}
\newcommand{\oo}{\infty}
\newcommand{\De}{\Delta}
\newcommand{\cd}{\cdot}
\newcommand{\df}{\partial}
\newcommand{\ndash}{\textendash}
\newcommand{\mdash}{\textemdash}

%>>>>> Аннотирование
\newcommand{\note}[2]{\overbrace{#1}^{#2}}% скобка сверху для комментария
% \overset{}{}% для указания символа над другим смиволом
% \underset{}{}% для указания символа под другим смиволом
%<<<<< Аннотирование

%>>>>>> Матрицы
\DeclareMathOperator{\rank}{rank}
\newcommand{\tvec}[1]{\mathbfit{#1}}% "text vector"
\newcommand{\mtx}[1]{\mathrm{#1}}
\newcommand{\transposed}[1]{{#1}^{\mathrm{T}}}
%>>>>>> Матрицы

%>>>>> Скобки
\newcommand{\lt}{\left}
\newcommand{\rt}{\right}
\newcommand{\la}{\langle}% '<'
\newcommand{\ra}{\rangle}% '>'
\newcommand{\avg}[1]{\langle{#1}\rangle}% '<X>'
%<<<<< Скобки

%>>>>> Дроби
\newcommand{\cf}[2]{\cfrac{#1}{#2}}
\newcommand{\fr}[2]{\frac{#1}{#2}}
%<<<<< Дроби


%>>>>> Стрелки
\newcommand{\Rarr}{\Rightarrow}% ⇒ следствие | лучше использовать \implies
\newcommand{\LRarr}{\Leftrightarrow}% равносильно | лучше  использовать \iff
\newcommand{\rarr}{\xrightarrow{}}% → стрелка вправо
\newcommand{\nwarr}{\nwarrow}% ↖ север-запад стрелка
\newcommand{\nearr}{\nearrow}% ↗ север-восток стрелка
\newcommand{\swarr}{\swarrow}% ↙ юг-запад стрелка
\newcommand{\searr}{\searrow}% ↘ юг-восток стрелка

\newcommand{\raises}{\nwarrow}% возрастает
\newcommand{\increases}{\nwarrow}% возрастает
\newcommand{\falls}{\swarrow}% убывает
\newcommand{\decreases}{\swarrow}% убывает

%{{{
\makeatletter
\newcommand{\impliesby}[2][]{\ext@arrow 0359\Leftrightarrowfill@{#1}{#2}}% следствие с надписью
\makeatother
%}}}

%{{{
\makeatletter
\newcommand{\iffby}[2][]{\ext@arrow 0359\Rightarrowfill@{#1}{#2}}% равносильность с надписью
\makeatother
%}}}
%<<<<< Стрелки

% Функции для удобного описания формул: https://tex.stackexchange.com/questions/95838/how-to-write-a-perfect-equation-parameters-description


%<<<<<< математические функции для удобства
%>>>>>> Стиль текста
\newcommand{\hex}[1]{\texttt{0{\footnotesize{x}}#1}}
\newcommand{\ttt}[1]{\texttt{#1}}
%<<<<<< Стиль текста

\newcommand\Chapter[3]{%
% Принимает 3 аргумента - название главы и дополнительный заголовок и множитель ширины загловка (можно ничего)
\refstepcounter{chapter}%
\chapter*{%
%\hfill % заполнение отступом пространства до заголовка
\begin{minipage}{#3\textwidth} % Можно изменить ширину министраницы (заголовка)
\flushleft % Выранивание заголовка по левому краю параграфа (заголовка)
%\flushright % Выранивание заголовка по правому краю параграфа (заголовка)
\begin{huge}%
% Отключена нумерация глав в тексте:
% \textbf{\chaptername\ \arabic{chapter}\\}
\textbf{#1}% Первый заголовок
\end{huge}%
\\% Перенос сторки
\begin{Huge}
#2% Второй заголовок
\end{Huge}
\end{minipage}
}%
% Отключена нумерация для chapter в toc (table of contents), т.е. Оглавлении (Содержании):
% \addcontentsline{toc}{chapter}{\arabic{chapter}. #1}
% Представление главы в содержании:
\addcontentsline{toc}{chapter}{#1. #2}%
}

\newcommand\Section[1]{
% Принимает 1 аргумент - название секции
\refstepcounter{section}
\section*{%
\raggedright
% Отключена дополнительная нумерация chapter в section в тексте документа:
% \arabic{chapter}.\arabic{section}. #1}
% Отключена любая нумарация section в тексте документа:
\arabic{section}. #1%
}

% Отключена дополнительная нумерация chapter в section в toc (table of contents) Оглавлении (Содержании):
% \addcontentsline{toc}{section}{\arabic{chapter}.\arabic{section}. #1}
\addcontentsline{toc}{section}{\arabic{section}. #1}
}


\newcommand\Subsection[1]{
% Принимает 1 аргумент - название подсекции
\refstepcounter{subsection}
\subsection*{%
\raggedright%
% Отключена дополнительная нумерация chapter в section в тексте документа (можно добавить отступ с помощью \hspace*{12pt}):
% \arabic{chapter}.\arabic{section}.\arabic{subsection}. #1}
\arabic{section}. \arabic{subsection}. #1
}
% Отключена дополнительная нумерация chapter в section в Оглавлении (Содержании):
%\addcontentsline{toc}{subsection}{\arabic{chapter}.\arabic{section}.\arabic{subsection}. #1}
\addcontentsline{toc}{subsection}{\arabic{subsection}. #1}
}


\newcommand\Figure[4]{
% Принимает 4 аргумента - название файла изображения, ее размер в тексте, описание, лэйбл (псевдоним в формате "fig:name")
%
\refstepcounter{figure}
\begin{figure}[H] %- \usepackage {float} %[h]
\begin{center}
\fbox{
\includegraphics[width=#2]{#1}
}
\end{center}
\begin{center}
Рис.~\arabic{figure}. #3.
\end{center}
%\caption{#3}
\label{fig:#4}
\end{figure}
}


\newcommand\Table[3]{
% Принимает 3 аргумента --- лэйбл name(#1) (псевдоним в формате "tab:name"), ее описание(#2), содержание таблицы(#3)
% ВАЖНО!: от этого способа страдает нумерация описаний, можно использовать создание таблиц через googlesheet
%
\renewcommand{\arraystretch}{1.2} % Установка высоты строки таблицы по умолчанию, увеличенное на 0.2 пункта
% \refstepcounter{table}% увеличение счетчика таблиц
\begin{table}[Htpb]% "right Here", "top", "new page", "bottom"
\label{tab:#1}% лэйбл таблицы, для ссылок
\resizebox{\columnwidth}{!}{% сжимает очень широкие таблицы, чтобы вместить на страницу
#3% Содержимое таблицы
}
%
\caption{#2}% Описание стандартными средствами для используемого окружения (table)
% \captionof{table}{#2}% Описание стандартными средствами
% \captionof*{figure}{\flushleft \textsc\textbf{Рис. 1.}}% Описание стандартными средствами, как рисунка
%
%%> кастомное описание
% \begin{flushleft}% Кастомное описание
%     % \textsf{%
%         \textbf{%
%             \\[2mm]
%             #2% Описание к картинке
%         }%
%         % \\[8mm]% Отступ
%     % }%
% \end{flushleft}
%%< кастомное описание
\end{table}
\renewcommand{\arraystretch}{1} % возврат установка высоты строки таблицы по умолчанию на 1
}


\newcommand\CustomFigure[4]{ % multicols не умеют в table и figure, поэтому приходится извращаться % вставка таблицы с меткой рисунка
% Принимает 4 аргумента - название файла изображения, ее размер в тексте, описание, лэйбл (псевдоним в формате "fig:name")
%
\refstepcounter{figure}
\begin{figure}[ht]% "here", "top"
\begin{center}
\includegraphics[width=#2]{#1}
\end{center}
%
%\caption{#3}
\captionof{figure}{#3}% описание стандартными средствами
% \begin{center}
\begin{flushleft} % Кастомное описание
\textbf{%
#3% Текст описания
}
\end{flushleft}
% \end{center}
%
\label{fig:#4}% Лэйбл, для ссылок
\end{figure}
}


\newcommand\CustomTableFigure[3]{% multicols не умеют в table и figure, поэтому приходится извращаться % вставка таблицы с меткой рисунка
%
% Принимает 3 аргумента --- лэйбл name(#1) (псевдоним в формате "tab:name"), ее описание(#2), содержание таблицы(#3)
%
\begin{center}
\refstepcounter{figure}
\label{tab:#1}% лэйбл таблицы, для ссылок
\resizebox{\columnwidth}{!}{% сжимает очень широкие таблицы, чтобы вместить на страницу
#3% Содержание таблицы
}
%
\captionof{figure}{#2}% Описание стандартными средствами
% \captionof*{figure}{\flushleft \textsc\textbf{Рис. 1.}}% Описание стандартными средствами
%
\begin{flushleft}% Кастомное описание
% \textsf{%
\textbf{%
\\[2mm]
#2% Описание к картинке
}%
% \\[8mm]% Отступ
% }%
\end{flushleft}
\end{center}
}


\newcommand{\InkscapeFigure}[4]{% Вставки иллюстраций из Inkscape (pdf+latex)
%
% Принимает 4 параметра: #1 название файла, #2 описание, #3 лейбл #4 размер
%
% \begin{minipage}{#4}
\begin{figure}[htbp]
\centering
\def\svgwidth{#4}
\import{./figures/}{#1.pdf_tex}
\caption{#2}
\label{fig:#3}
\end{figure}
% \end{minipage}
}


\newcommand\Equation[3]{% Кастомное оформление выражений
%
% Принимает 3 аргумента --- лэйбл name (#1) (псевдоним в формате "tab:name"), его описание(#2), содержание выражения (#3)
%
\textbf{#2}% описание
\begin{equation}
#3% содержимое выражений
\label{eq:#1}% лэйбл
\end{equation}
}

%<<<<<<<<<<<<<<<<<<<<<<<<<<<< ДОПОЛНИТЕЛЬНЫЕ КОМАНДЫ <<<<<<<<<<<<<<<<<<<<<<<<<<
%<<<<<<<<<<<<<<<<<<<<<< ПРЕАМБУЛА <<<<<<<<<<<<<<<<<<<<<<<<<


%%%%%%%%%%%%%%%%%%% СОДЕРЖИМОЕ ОТЧЕТА %%%%%%%%%%%%%%%%%%%%%
%>>>>>>>>>>>>>>> ''''''''''''''''''''''' >>>>>>>>>>>>>>>>>>
\begin{document}


%>>>>>>>>>>>>>>>> ОПРЕДЕЛЕНИЕ НАЗВАНИЙ >>>>>>>>>>>>>>>>>>>>
% Переоформление некоторых стандартных названий
%\renewcommand{\chaptername}{Лабораторная работа}
    \renewcommand{\chaptername}{\lab\ \labnumber} % переименование глав
    \renewcommand{\contentsname}{Содержание} % переименование оглавления
%<<<<<<<<<<<<<<<< ОПРЕДЕЛЕНИЕ НАЗВАНИЙ <<<<<<<<<<<<<<<<<<<<
% \setlength{\itemsep}{0pt} % установка расстояния между строчками в списках можно использовать локально внутри списка списке
% \setlength{\parskip}{0pt} % 
% \setlength{\parsep}{0pt}  % 

%>>>>>>>>>>>>>>>>> ТИТУЛЬНАЯ СТРАНИЦА >>>>>>>>>>>>>>>>>>>>>
    \include{titlepage}
%<<<<<<<<<<<<<<<<< ТИТУЛЬНАЯ СТРАНИЦА <<<<<<<<<<<<<<<<<<<<<


%>>>>>>>>>>>>>>>>>>>>> СОДЕРЖАНИЕ >>>>>>>>>>>>>>>>>>>>>>>>>
% Содержание
    \tableofcontents
%<<<<<<<<<<<<<<<<<<<<< СОДЕРЖАНИЕ <<<<<<<<<<<<<<<<<<<<<<<<<


%%%%%%%%%%%%%%%%%%%%%%% КОД РАБОТЫ %%%%%%%%%%%%%%%%%%%%%%%%
%>>>>>>>>>>>>>>>>>>>'''''''''''''''''>>>>>>>>>>>>>>>>>>>>>
    \newpage
    \Chapter{\lab\ \labnumber}{}{}

    \Section{Задание варианта \variant}

    \noindent
%    \textbf{
%    % Заглавное описание....:
%        Заголовок
%    }
%
%    \textit{
%    % Описание задания...
%        Описание
%    }

 На основе базового класса \verb|Pokemon| написать свои классы для заданных видов покемонов.
    Каждый вид покемона должен иметь один или два типа и стандартные базовые характеристики:
    \begin{itemize}
        \item очки здоровья (HP)
        \item атака (attack)
        \item защита (defense)
        \item специальная атака (special attack)
        \item специальная защита (special defense)
        \item скорость (speed)
    \end{itemize}

    Классы покемонов должны наследоваться в соответствии с цепочкой эволюции покемонов.
    На основе базовых классов \verb|PhysicalMove|,\verb|SpecialMove| и \verb|StatusMove| реализовать свои классы для заданных видов атак.

    Атака должна иметь стандартные тип, силу (power) и точность (accuracy).
    Должны быть реализованы стандартные эффекты атаки.
    Назначить каждому виду покемонов атаки в соответствии с вариантом.
    Уровень покемона выбирается минимально необходимым для всех реализованных атак.

    Используя класс симуляции боя \verb|Battle|, создать 2 команды покемонов (каждый покемон должен иметь имя) и запустить бой.

    Базовые классы и симулятор сражения находятся в \href{https://se.ifmo.ru/documents/10180/660917/Pokemon.jar/a7ce60af-6ee6-47d0-a95e-e5ed9a697bd2}{jar-архиве}(обновлен 9.10.2018, исправлен баг с добавлением атак и кодировкой).
    Документация в формате javadoc - \href{https://se.ifmo.ru/~tony/doc/}{здесь}.

    Информацию о покемонах, цепочках эволюции и атаках можно найти на сайтах \href{https://poke-universe.ru/}{https://poke-universe.ru}, \href{https://pokemondb.net/}{https://pokemondb.net},\href{https://veekun.com/dex/pokemon}{https://veekun.com/dex/pokemon}

    \subsubsection*{Комментарии}
    Цель работы: на простом примере разобраться с основными концепциями ООП и научиться использовать их в программах.

    Что надо сделать (краткое описание)

    \begin{enumerate}
        \setlength{\itemsep}{0pt} % Сокращение межстрочных расстояний
        \setlength{\parskip}{0pt}
        \setlength{\parsep}{0pt}
        \item Ознакомиться с \href{https://se.ifmo.ru/~tony/doc/}{документацией}, обращая особое внимание на классы \verb|Pokemon| и \verb|Move|.
        При дальнейшем выполнении лабораторной работы читать документацию еще несколько раз.
        \item Скачать файл Pokemon.jar.
        Его необходимо будет использовать как для компиляции, так и для запуска программы.
        Распаковывать его не надо!
        Нужно научиться подключать внешние jar-файлы к своей программе.
        \item Написать минимально работающую программу и посмотреть как она работает.
        \lstinputlisting[label={lst:java1},language=Java]{../example/MainExample.java}
        \item Создать один из классов покемонов для своего варианта.
        Класс должен наследоваться от базового класса \verb|Pokemon|.
        В конструкторе нужно будет задать типы покемона и его базовые характеристики.
        После этого попробуйте добавить покемона в сражение.
        \item Создать один из классов атак для своего варианта (лучше всего начать с физической или специальной атаки).
        Класс должен наследоваться от класса \verb|PhysicalMove| или \verb|SpecialMove|.
        В конструкторе нужно будет задать тип атаки, ее силу и точность.
        После этого добавить атаку покемону и проверить ее действие в сражении.
        Не забудьте переопределить метод \verb|describe|, чтобы выводилось нужное сообщение.
        \item Если действие атаки отличается от стандартного, например, покемон не промахивается, либо атакующий покемон также получает повреждение, то в классе атаки нужно дополнительно переопределить соответствующие методы (см.
        документацию).
        При реализации атак, которые меняют статус покемона (наследники \verb|StatusMove|), скорее всего придется разобраться с классом \verb|Effect|.
        Он позволяет на один или несколько ходов изменить состояние покемона или модификатор его базовых характеристик.
        \item Доделать все необходимые атаки и всех покемонов, распределить покемонов по командам, запустить сражение.
    \end{enumerate}
    \begin{figure}[H] % 'H' -- вставить тут же (подключен модуль), обычный вариант: 'htpb'
        \centering
        % { граница для иллюстрации
        % \setlength{\fboxsep}{0pt}% убрать отсутп от границы
        % \setlength{\fboxrule}{1pt}%
        % \fbox{%
        \includegraphics[width=\textwidth]{pokemons.png}
        % }} % ограничение области действия параметров
        \caption{Покемоны и их атаки}
        \label{fig:enter-label}
    \end{figure}


    \newpage
    \Section{Выполнение задания.}
Задание было выполнено в редакторе кода Visual Studio Code и собрано в \verb|jar| файл \verb|lab2.jar|. Полный исходный код был загружен в Git репозиторий на GitHub \cite{github}.
%    \newpage
    \Subsection{Листинги кода}
    Листинг из файла~\ref{lst:java}
    \lstinputlisting[caption={Исходный код главного класса программы},label={lst:java},language=Java]{../src/Main.java}

%    Листинг в код latex \ref{lst:sql}
%    \begin{lstlisting}[caption={SQL},label={lst:sql}]
%declare @t table(
%  id int
%)
%    \end{lstlisting}

%    Листинг прямо в текст: \lstinline[columns=fixed]{declare}. Либо еще так: \verb|declare|.

%    Вставленное изображение с описанием и шириной по тексту.
    \begin{figure}[H] % 'H' -- вставить тут же (подключен модуль), обычный вариант: 'htpb'
        \centering
        % { граница для иллюстрации
        % \setlength{\fboxsep}{0pt}% убрать отсутп от границы
        % \setlength{\fboxrule}{1pt}%
        % \fbox{%
        \includegraphics[width=\textwidth]{project_diagram.png}
        % }} % ограничение области действия параметров
        \caption{UML диаграмма классов с методами и полями}
        \label{fig:enter-label2}
    \end{figure}


% Выполнение задания...
    \Section{Результат работы программы.}

        \begin{lstlisting}[caption={Результат выполнения программы},label={lst:result}]
Numel Numel from the team Greren enters the battle!
Azelf Azelf from the team White enters the battle!
Azelf Azelf is using Energy Ball. The pokemon lowers the target’s Special Defense by one stage. 
Numel Numel loses 6 hit points.

Numel Numel is using Flame charge. The pokemon deals damage and raises its Speed by one stage. 
Azelf Azelf loses 8 hit points.
Numel Numel increases speed.

Azelf Azelf misses

Numel Numel is using Flame charge. The pokemon deals damage and raises its Speed by one stage. 
Azelf Azelf loses 6 hit points.
Numel Numel increases speed.
Azelf Azelf faints.
Clefairy Clefairy from the team White enters the battle!
Numel Numel is using Rest. The pokemon sleeps for two turns, completely healing itself.. 
Numel Numel is sleeping
Numel Numel restores 6 hit points.

Clefairy Clefairy is using Growl. The pokemon lowers the target's Attack by one stage. 
Numel Numel decreases attack.

Numel Numel is using Flame charge. The pokemon deals damage and raises its Speed by one stage. 
Clefairy Clefairy loses 5 hit points.
Numel Numel increases speed.

Clefairy Clefairy is using Growl. The pokemon lowers the target's Attack by one stage. 
Numel Numel decreases attack.

Numel Numel is using Flame charge. The pokemon deals damage and raises its Speed by one stage. 
Clefairy Clefairy restores 1 hit points.
Numel Numel increases speed.

Clefairy Clefairy is using Growl. The pokemon lowers the target's Attack by one stage. 
Numel Numel decreases attack.

Numel Numel is using Rest. The pokemon sleeps for two turns, completely healing itself.. 
Numel Numel is sleeping

Clefairy Clefairy is using Growl. The pokemon lowers the target's Attack by one stage. 
Numel Numel decreases attack.

Numel Numel is using Rest. The pokemon sleeps for two turns, completely healing itself.. 

Clefairy Clefairy is using Growl. The pokemon lowers the target's Attack by one stage. 
Numel Numel decreases attack.

Numel Numel is using Rest. The pokemon sleeps for two turns, completely healing itself.. 
Numel Numel is sleeping

Clefairy Clefairy is using Growl. The pokemon lowers the target's Attack by one stage. 
Numel Numel decreases attack.

Numel Numel misses

Clefairy Clefairy is using Growl. The pokemon lowers the target's Attack by one stage. 
Numel Numel decreases attack.

Numel Numel is using Flame charge. The pokemon deals damage and raises its Speed by one stage. 
Clefairy Clefairy loses 7 hit points.
Numel Numel increases speed.

Clefairy Clefairy is using Growl. The pokemon lowers the target's Attack by one stage. 
Numel Numel decreases attack.

Numel Numel misses

Clefairy Clefairy is using Growl. The pokemon lowers the target's Attack by one stage. 
Numel Numel decreases attack.

Numel Numel is using Rest. The pokemon sleeps for two turns, completely healing itself.. 
Numel Numel is sleeping

Clefairy Clefairy is using Growl. The pokemon lowers the target's Attack by one stage. 
Numel Numel decreases attack.

Numel Numel misses

Clefairy Clefairy is using Growl. The pokemon lowers the target's Attack by one stage. 
Numel Numel decreases attack.

Numel Numel is using Rest. The pokemon sleeps for two turns, completely healing itself.. 
Numel Numel is sleeping

Clefairy Clefairy is using Growl. The pokemon lowers the target's Attack by one stage. 
Numel Numel decreases attack.

Numel Numel misses

Clefairy Clefairy is using Growl. The pokemon lowers the target's Attack by one stage. 
Numel Numel decreases attack.

Numel Numel misses

Clefairy Clefairy is using Growl. The pokemon lowers the target's Attack by one stage. 
Numel Numel decreases attack.

Numel Numel is using Flame charge. The pokemon deals damage and raises its Speed by one stage. 
Clefairy Clefairy loses 6 hit points.
Numel Numel increases speed.
Clefairy Clefairy faints.
Camerupt Camerupt from the team White enters the battle!
Numel Numel is using Rest. The pokemon sleeps for two turns, completely healing itself.. 
Numel Numel is sleeping

Camerupt Camerupt is using Rock Polish. The pokemon raises its Speed by two stages. 
Camerupt Camerupt increases speed.

Numel Numel is using Flame charge. The pokemon deals damage and raises its Speed by one stage. 
Camerupt Camerupt loses 3 hit points.
Numel Numel increases speed.

Camerupt Camerupt is using Rock Polish. The pokemon raises its Speed by two stages. 
Camerupt Camerupt increases speed.

Numel Numel misses

Camerupt Camerupt is using Rock Polish. The pokemon raises its Speed by two stages. 
Camerupt Camerupt increases speed.

Camerupt Camerupt is using Rock Polish. The pokemon raises its Speed by two stages. 
Camerupt Camerupt increases speed.

Numel Numel is using Flame charge. The pokemon deals damage and raises its Speed by one stage. 
Camerupt Camerupt loses 3 hit points.
Numel Numel increases speed.

Camerupt Camerupt is using Rock Polish. The pokemon raises its Speed by two stages. 
Camerupt Camerupt increases speed.

Numel Numel misses

Camerupt Camerupt is using Rock Polish. The pokemon raises its Speed by two stages. 
Camerupt Camerupt increases speed.

Numel Numel misses

Camerupt Camerupt is using Rock Polish. The pokemon raises its Speed by two stages. 
Camerupt Camerupt increases speed.

Numel Numel is using Rest. The pokemon sleeps for two turns, completely healing itself.. 
Numel Numel is sleeping

Camerupt Camerupt is using Rock Polish. The pokemon raises its Speed by two stages. 
Camerupt Camerupt increases speed.

Numel Numel misses

Camerupt Camerupt is using Rock Polish. The pokemon raises its Speed by two stages. 
Camerupt Camerupt increases speed.

Numel Numel misses

Camerupt Camerupt is using Rock Polish. The pokemon raises its Speed by two stages. 
Camerupt Camerupt increases speed.

Numel Numel is using Rest. The pokemon sleeps for two turns, completely healing itself.. 
Numel Numel is sleeping

Camerupt Camerupt is using Rock Polish. The pokemon raises its Speed by two stages. 
Camerupt Camerupt increases speed.

Numel Numel is using Flame charge. The pokemon deals damage and raises its Speed by one stage. 
Camerupt Camerupt loses 3 hit points.
Numel Numel increases speed.

Camerupt Camerupt is using Rock Polish. The pokemon raises its Speed by two stages. 
Camerupt Camerupt increases speed.

Numel Numel is using Flame charge. The pokemon deals damage and raises its Speed by one stage. 
Camerupt Camerupt loses 3 hit points.
Numel Numel increases speed.

Camerupt Camerupt is using Rock Polish. The pokemon raises its Speed by two stages. 
Camerupt Camerupt increases speed.

Numel Numel is using Flame charge. The pokemon deals damage and raises its Speed by one stage. 
Camerupt Camerupt loses 2 hit points.
Numel Numel increases speed.
Camerupt Camerupt faints.
Team White loses its last Pokemon.
The team Greren wins the battle!

        \end{lstlisting}

    \newpage

    \Section{Вывод}
% Вывод...
    Во время выполнения лабораторной работы я изучил принципы ООП, научился импортировать \verb|jar| файлы как библиотеки, научился расширять классы и рабоать с модификаторами доступа. Также в процессе выполения я тесно работал с документацией\cite{sedoc} на библиотеку для покемонов и сторонними сайтами для поиска информации о покемонах (PokemonDB\cite{pokemondb}). Полученные мною знания являются необходимой базой для дальнейшего изучения языка и разработки уже более комлпексных проектов.\\
    Помимо этого, я ознакомился с способами создания UML диаграм и сделал диаграмму на практике при помощи PlantUML.
% -- из biblist
    \newpage
%<<<<<<<<<<<<<<<<<<<<<< КОД РАБОТЫ <<<<<<<<<<<<<<<<<<<<<<<<


%>>>>>>>>>>>>>>>> СПИСОК ЛИТЕРАТУРЫ >>>>>>>>>>>>>>>>>>>>>>>
    \begin{thebibliography}{}
    \bibitem{github} Cсылка на личный репозиторий GitHub: \url{https://github.com/DaniilRen/ITMO-labs/tree/main/Java/Lab3}\\
    \bibitem{pokemondb} Ссылка на сайт с информацией о покемонах: \url{https://pokemondb.net}\\
    \bibitem{sedoc} Ссылка на документацию по jar библиотеке с покемонами: \url{https://se.ifmo.ru/~tony/doc/}\\
\end{thebibliography}  % Для соответсвия гост, придется доработать. Нужен файл .bib
%<<<<<<<<<<<<<<<<<<<< СПИСОК ЛИТЕРАТУРЫ <<<<<<<<<<<<<<<<<<<


\end{document}
%<<<<<<<<<<<<<<<< ,,,,,,,,,,,,,,,,,,,,,,, <<<<<<<<<<<<<<<<<
%<<<<<<<<<<<<<<<<<<< СОДЕРЖИМОЕ ОТЧЕТА <<<<<<<<<<<<<<<<<<<<