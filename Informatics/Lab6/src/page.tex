\begingroup
\fontsize{9pt}{10pt}\selectfont

\newgeometry{top=1cm, bottom=1.8cm, left=1cm, right=1.75cm} 
\setcounter{page}{3}
\graphicspath{{assets}}
\setlength{\parindent}{6pt}
\begin{center}
  \scriptsize{\so{СТАТИСТИКА ПЕРВЫХ ЦИФР СТЕПЕНЕЙ ДВОЙКИ И ПЕРЕДЕЛ МИРА}}
	\vspace{-2ex}
\end{center}

\begin{multicols*}{3}
	\fontsize{9pt}{10pt}\selectfont
	\begin{figure}[H]
		\includegraphics[width=\columnwidth]{image-1.png}
		\caption*{Рис. 1. Траектория точки при итерациях поворота окружности}
		\vspace{-1ex}
	\end{figure}
	\par{
	\noindent
	бой дуге окружности, асимтотически
	(при большом времени наблюдения)
	пропорционально длине дуги (и не
	зависит ни от положения дуги на
	окружности, ни от начальной точки,
	ни даже от величины угла поворота).}
	\par{Распределение первых цифр чисел
	$2^n$ получается теперь следующим об
	разом. Рассмотрим последовательность чисел $ \lg(2^n) = nx $. Число $ x = \lg(2) $
	иррационально (здесь и далее $\lg$ - логарифм по основанию 10). По
	теореме последовательность дробных
	долей чисел их равномерно распреде
	лена на интервале (0, 1).}
	\par{Но первая цифра і числа $ 2^n $ определяется тем, в какой из интервалов
	между числами $ \lg(i + 1) $ и $ \lg(i) $ nonaдает дробная доля числа $ \lg(2^n) $. По теореме, доли чисел $ 2^n $, начинающихся с i ( = 1, 2,... 9) составляют
	$ p_i = \lg(i + 1) - \lg i $. Hапример, для первой цифры $ і = 1 $ эта доля составляет $ \lg(2) = 0,301 $ (близость этого логарифма к $ \frac{3}{10} $ отражает близость $ 2^10 = 1024 $ к $ 1000 - 10^3 $). 
	Поэтому доля единиц среди первых цифр чисел $ 2^n $ составляет примерно 30\% . Доли всех цифр (в процентах) даются таблицей 1.}
	

	\vspace{-\baselineskip}


	\begin{table}[H]
		\caption*{\tiny{Таблица 1}}
		\vspace{-0.25cm}
		\centering
		\fontsize{9pt}{10pt}\selectfont
		\begin{tabularx}{\columnwidth}{|X|c|c|c|c|c|}
			\hline
			$i$        & 1  & 2  & 3  & 4  & 5  \\
			\hline
			100 $p.$   & 30 & 17 & 12 & 10 & 8  \\
			\hline
			$i$        & 6  & 7  & 8  & 9  &    \\
			\hline
			100 $p.$   & 7  & 6  & 5  & 5  &    \\
			\hline
		\end{tabularx}
	\end{table}

	\par{	Девяток примерно в 6 раз меньше, чем единиц (рис.2).}
	\begin{figure}[H]
		\includegraphics[width=\columnwidth]{image-2.png} % Insert your image
		\caption*{Рис. 2. Распределение первых цифр степеней двойки}
	\end{figure}
	\columnbreak

	\par{Из всего сказанного для дальнейшего важен такой вывод: приведенное в таблице странно неравномерное распределение первых цифр чисел последовательности $ 2^n $ объясняется равномерным распределением дробных долей логарифмов чисел нашей
	последовательности.} 
	\par{Этот вывод приводит к одинаковому распределению первых цифр для многих различных последовательностей (например, для геометрических прогрессий $ 2^n $ или $ 3^n $, но не только для них).}

	{\raggedright
  \noindent\rule{\linewidth}{0.4pt}
  \vspace{-0.45cm}
  {\large Население стран мира}\par
  \noindent\rule{\linewidth}{0.4pt}
	}
	\noindent
	Лет двадцать назад Н.Н. Константинов сообщил мне, что первые цифры населений стран мира распределены так же, как первые цифры степеней двойки (табл. 2).
	\begin{table}[H]
		\caption*{\tiny{Таблица 2}}
		\vspace{-0.25cm}
		\centering
		\fontsize{9pt}{10pt}\selectfont
		\begin{tabularx}{\columnwidth}{|X|c|c|c|c|c|}
			\hline
			первая цифра & 1  & 2  & 3  & 4  & 5  \\
			\hline
			процент числа стран, 1995 & 29 & 21 & 10 & 11 & 6  \\
			\hline
			первая цифра & 6  & 7  & 8  & 9  &    \\
			\hline
			процент числа стран, 1995 & 6 & 8 & 3 & 6 &    \\
			\hline
		\end{tabularx}
	\end{table}
	\noindent
	Вот мое тогдашнее объяснение этого факта.

	\par{Согласно теории Мальтуса, насе ление каждой страны растет в геометрической прогрессии. Из теоремы Вейля (см. предыдущий раздел) следует, что первые цифры населения
	фиксированной страны в последовательные годы распределены как пер вые цифры степеней двойки (см. рис. 2). Согласно «эргодической теореме» (или, лучше сказать, согласноэргодическому принципу), временвое среднее можно заменить пространственным: распределение по
	странам в один и тот же год должно совпадать с распределением в одной стране в разные годы.}
	\par{Для контроля теории я рассмотрел числа страниц в книгах моей библиотеки, длины рек и высоты гор. Во всех этих случаях доли единиц и доли девяток среди первых цифр
	полученных чисел оказались практически одинаковыми: $ p_i = \frac{1}{9} $. Книги, реки и горы не растут в геометрической прогрессии, теория Мальтуса к ним не применима. Поэтому разли-}

	\columnbreak
	\noindent
	чие статистик первых цифр в числах,
	выражающих населения и, скажем,
	длины рек, служит своеобразным
	косвенным подтверждением форму
	лы Мальтуса (согласно которой на
	селение растет в геометрической прогрессии).
	\par{Однако лет десять назад М.Б. Сев
	рюк обнаружил, что не только насе
	ления, но и площади стран мира
	подчиняются такому же закону распределения первых цифр, как степени двойки! К площадям теория Маль
	туса, по-видимому, неприменима, так
	что возник вопрос - как объяснить
	это поведение площадей. Ниже я
	пытаюсь дать ответ на этот вопрос.}
	
	
	{\raggedright
		\noindent\rule{\linewidth}{0.4pt}
		\vspace{-0.18cm}
		{\large Площади стран мира}\par
		\noindent\rule{\linewidth}{0.4pt}
	}

	\noindent
	Предыдущие примеры подсказывают, что следует искать причину странного распределения первых цифр
	площадей стран мира либо в их росте, либо в убывании (в геометрической прогрессии). История мира показывает, что площади стран (особенно империй) иногда растут, а иног
	да убывают за счет то присоединения одних стран к другим, то распада.
	Рассмотрим вначале самую примитивную модель этого явления. Предположим, что за единицу времени страна с вероятностью половина делится пополам, а с вероятностью половина присоединяет к себе другую
	страну такой же площади, как она сама,
	\par{	\textbf{Теорема}. \textit{Распределение дробных долей логарифмов площадей, занимаемых такой случайной страной в
	момент п, стремится к равномерному распределению на интервале (0, 1) при п, стремящемся к бесконечности}.}
	\par{Иными словами, вероятность того, что первая цифра площади окажется единицей, стремится при по к $ \lg(2) = 0,301, ..., $ что она окажется девяткой - примерно к 0,046.}
	\par{Действительно, рассмотрим последовательность $ 12 = \lg(S(n)) $, где S -площадь в момент п. Точка 4, в
	следующий момент $ n+1 $ с одинаковой вероятностью сдвигается влево или вправо на $ \lg(2) $ (причем, конечно, выбор, что делать - делиться
	или объединяться, — в каждый момент времени независим от выбора в другие моменты времени). По законам теории вероятностей, распределение величины $ l_n $, при больших $ n $
	будет в основном сосредоточено на}
	


\end{multicols*}

\endgroup