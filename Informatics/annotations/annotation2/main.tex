%2025_Шаблон аннотации (.tex)
\documentclass[12pt]{article}
\pagestyle{empty}
\usepackage[utf8]{inputenc}
\usepackage[russian]{babel}
\usepackage[top=1cm,left=1cm,right=2cm, bottom=1cm]{geometry}
\usepackage{tabularx}% http://ctan.org/pkg/tabularx
%\usepackage{hyperref}
\begin{document}

\begin{center}
\quad Университет ИТМО, факультет программной инженерии и компьютерной техники \\
\quad Двухнедельная отчётная работа по «Информатике»: аннотация к статье\\
\end{center}

\begin{tabular}{
  |p{\dimexpr.1\linewidth-2\tabcolsep-1.3333\arrayrulewidth}% column 1
  |p{\dimexpr.1\linewidth-2\tabcolsep-1.3333\arrayrulewidth}% column 2
  |p{\dimexpr.47\linewidth-2\tabcolsep-1.3333\arrayrulewidth}% column 3
  |p{\dimexpr.13\linewidth-2\tabcolsep-1.3333\arrayrulewidth}% column 4
  |p{\dimexpr.1\linewidth-2\tabcolsep-1.3333\arrayrulewidth}% column 5
  |p{\dimexpr.1\linewidth-2\tabcolsep-1.3333\arrayrulewidth}|% column 6
  }
  \hline
  \centering\small{Дата прошедшей лекции} & \centering\small{Номер прошедшей лекции} & \centering\small{Название статьи/главы книги/видеолекции} & \centering\small{Дата публикации} & \centering\small{Размер статьи} & \centering\arraybackslash \small{Дата сдачи} \\ 
 \hline
 \centering 24.09.25 & \centering 2 & Метод регенерационного блочного кодирования & \centering 31.03.2025 & \centering 2124 & 13.10.23  \\
  \hline

\hline
\end{tabular}

\begin{center}
\quad Выполнил(а) \underline{\hspace{1.5cm}Бых Даниил Максимович\hspace{1.5cm}}, № группы \underline{ P3109 }, оценка \underline{\hspace{2cm}}
\end{center}

\begin{tabularx}{\textwidth} { 
	| >{\raggedright\arraybackslash}X|} \hline
		\textbf{Прямая полная ссылка на источник или сокращённая ссылка} \\
		\url{https://cyberleninka.ru/article/n/metod-regeneratsionnogo-blochnogo-kodirovaniya/viewer}
		\smallskip\\
		\hline
		\textbf{Теги, ключевые слова или словосочетания}\\
		Помехоустойчивое кодирование, кодирование, избыточность, акустический корпус, акустический канала связи.
		\smallskip\\
		\hline
		\textbf{Перечень фактов, упомянутых в статье}\\
		1) Акустический канал позволяет обмениваться информацией как по воздуху, так и по проводу. \\
		2) Не существует правовых ограничений в рамках существующих интерфейсов ввода-вывода информации по акустическому каналу на различных устройствах от различных производителей. \\
		3) Следствием воздействия шумов становится потеря целых блоков передаваемой информации. \\
		4) Взаимосвязь музыкальной гармонии и ритма позволяет позволяет восстановливать целостность сообщения.\\
		5) Существует 2 основных типа помехоустойчивого кодирования информации: добавление избыточной информации и преобразование сообщения в определенный формат данных. \\
		6) В сравнении с другими помехоустойчивыми кодами (линейными блочными, сверточными, кодом Рида-Соломона), регенеративный метод отличается отсутствием избыточности и возможностью передачи метаданных в структуре кода.
		\hline
		\textbf{Позитивные следствия и/или достоинства описанной в статье технологии}\\
		1) Выполнение регенерации блоков потерянной информации без внесения избыточности в структуру исходного сообщения. \\
		2) Восстановления потерянных блоков сообщения произвольного объема. \\
		3) Возможность передачи вместе с исходным сообщением различных метаданных в структуре связей транспортировочного корпуса. \\
		\hline
		\textbf{Негативные следствия и/или достоинства описанной в статье технологии}\\
		1) Высокая сложность реализации и вычислительных затрат при внедрении в существующие системы. \\
		2) Сложность настройки и параметризации для достижения оптимальной эффективности алгоритма передачи данных. \\
		3) Зависимость эффективности метода от качества канала связи. \\
		\hline
		\textbf{Ваши замечания, пожелания преподавателю или анекдот о программистах}\\
		HTML - язык программирования.
		\bigskip\\
		\hline
		
\end{tabularx}


\end{document}
