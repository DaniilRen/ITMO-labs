%2025_Шаблон аннотации (.tex)
\documentclass[12pt]{article}
\pagestyle{empty}
\usepackage[utf8]{inputenc}
\usepackage[russian]{babel}
\usepackage[top=1cm,left=1cm,right=2cm, bottom=1cm]{geometry}
\usepackage{tabularx}% http://ctan.org/pkg/tabularx
%\usepackage{hyperref}
\begin{document}

\begin{center}
\quad Университет ИТМО, факультет программной инженерии и компьютерной техники \\
\quad Двухнедельная отчётная работа по «Информатике»: аннотация к статье\\
\end{center}

\begin{tabular}{
	|p{\dimexpr.1\linewidth-2\tabcolsep-1.3333\arrayrulewidth}% column 1
	|p{\dimexpr.1\linewidth-2\tabcolsep-1.3333\arrayrulewidth}% column 2
	|p{\dimexpr.47\linewidth-2\tabcolsep-1.3333\arrayrulewidth}% column 3
	|p{\dimexpr.13\linewidth-2\tabcolsep-1.3333\arrayrulewidth}% column 4
	|p{\dimexpr.1\linewidth-2\tabcolsep-1.3333\arrayrulewidth}% column 5
	|p{\dimexpr.1\linewidth-2\tabcolsep-1.3333\arrayrulewidth}|% column 6
	}
	\hline
	\centering\small{Дата прошедшей лекции} & \centering\small{Номер прошедшей лекции} & \centering\small{Название статьи/главы книги/видеолекции} & \centering\small{Дата публикации} & \centering\small{Размер статьи} & \centering\arraybackslash \small{Дата сдачи} \\ 
	\hline
	\centering 10.09.25 & \centering 1 & Использование изменения системы отсчета для улучшения результатов анализа с помощью закона бенфорда & \centering 31.03.2025 & \centering 1208 & 24.09.25  \\
	\hline
	\centering 24.09.25 & \centering 2 & Метод регенерационного блочного кодирования & \centering 13.10.2023 & \centering 3356 & 08.10.25  \\
	\hline
	\centering 08.10.25 & \centering 3 & Многоцелевая фильтрация текста с использованием регулярных выражений & \centering 10.06.2024 & \centering 2124 & 05.11.25  \\
	\hline
	\centering 22.10.25 & \centering 4 & Формат PARQUET: Обзор структуры и методы оптимизации для высокой производительности в распределённых системах & \centering 11.04.2025 & \centering 3651 & 05.11.25  \\
	\hline
	\centering 05.11.25 & \centering 5 & Классификация определений в математических LaTeX статьях & \centering 17.05.2024 & \centering 3471 & 22.11.25 \\
	\hline
	\centering 19.11.25 & \centering 6 & Способ синтеза интерактивных образовательных ресурсов формата PDF с применением LuaLaTeX & \centering 11.02.2023 & \centering 2547 & 22.11.25 \\
	\hline
	\centering 03.12.25 & \centering 7 & Разработка через тестирование (TDD) & \centering 10.08.2023 & \centering 1676 & 18.12.25 \\
\hline

\hline
\end{tabular}

\begin{center}
\quad Выполнил(а) \underline{\hspace{1.5cm}Бых Даниил Максимович\hspace{1.5cm}}, № группы \underline{ P3109 }, оценка \underline{\hspace{2cm}}
\end{center}

\begin{tabularx}{\textwidth} { 
	| >{\raggedright\arraybackslash}X|} \hline
		\textbf{Прямая полная ссылка на источник или сокращённая ссылка} \\
		\url{https://cyberleninka.ru/article/n/razrabotka-cherez-testirovanie-tdd/viewer}
		\smallskip\\
		\hline
		\textbf{Теги, ключевые слова или словосочетания}\\
		tdd, разработка через тестировение, рефакторинг, методология разработки
		\smallskip\\
		\hline
		\textbf{Перечень фактов, упомянутых в статье}\\
		1) TDD (Test Driven Development, «разработка через тестирование») - методология, при которой сначала пишутся тесты на новый функционал, а затем код. \\
		2) Тестирование может быть ручным и автоматизированным. \\
		3) Цикл TDD: создать тест для нового функционала, убедиться, что тест падает, написать реализацию, снова запустить тесты.\\
		4) «Чистый тест» независим от других тестов и быстро воспроизводимость в любой среде. \\
		\hline
		\textbf{Позитивные следствия и/или достоинства описанной в статье технологии}\\
		1) Повышение стабильности и качества кода. \\
		2) Упрощение поддержки и рефакторинга. \\
		3) Сокращение объёма ручного тестирования и времени на тестирование в целом за счёт автоматизации. \\
		\hline
		\textbf{Негативные следствия и/или достоинства описанной в статье технологии}\\
		1) TDD применим не всегда. \\
		2) Для применения TDD требуется дополнительное время и усилия на написание и поддержку тестов,. \\
		3) Есть риск, что тесты будут подстраиваться не под реальные требования, а под текущую реализацию. \\
		\hline
		\textbf{Ваши замечания, пожелания преподавателю или анекдот о программистах}\\
		С наступающим, Павел Валерьевич!
		\bigskip\\
		\hline
		
\end{tabularx}


\end{document}
