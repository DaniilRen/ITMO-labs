%2025_Шаблон аннотации (.tex)
\documentclass[12pt]{article}
\pagestyle{empty}
\usepackage[utf8]{inputenc}
\usepackage[russian]{babel}
\usepackage[top=1cm,left=1cm,right=2cm, bottom=1cm]{geometry}
\usepackage{tabularx}% http://ctan.org/pkg/tabularx
%\usepackage{hyperref}
\begin{document}

\begin{center}
\quad Университет ИТМО, факультет программной инженерии и компьютерной техники \\
\quad Двухнедельная отчётная работа по «Информатике»: аннотация к статье\\
\end{center}

\begin{tabular}{
	|p{\dimexpr.1\linewidth-2\tabcolsep-1.3333\arrayrulewidth}% column 1
	|p{\dimexpr.1\linewidth-2\tabcolsep-1.3333\arrayrulewidth}% column 2
	|p{\dimexpr.47\linewidth-2\tabcolsep-1.3333\arrayrulewidth}% column 3
	|p{\dimexpr.13\linewidth-2\tabcolsep-1.3333\arrayrulewidth}% column 4
	|p{\dimexpr.1\linewidth-2\tabcolsep-1.3333\arrayrulewidth}% column 5
	|p{\dimexpr.1\linewidth-2\tabcolsep-1.3333\arrayrulewidth}|% column 6
	}
	\hline
	\centering\small{Дата прошедшей лекции} & \centering\small{Номер прошедшей лекции} & \centering\small{Название статьи/главы книги/видеолекции} & \centering\small{Дата публикации} & \centering\small{Размер статьи} & \centering\arraybackslash \small{Дата сдачи} \\ 
	\hline
	\centering 10.09.25 & \centering 1 & Использование изменения системы отсчета для улучшения результатов анализа с помощью закона бенфорда & \centering 31.03.2025 & \centering 1208 & 24.09.25  \\
	\hline
	\centering 24.09.25 & \centering 2 & Метод регенерационного блочного кодирования & \centering 13.10.2023 & \centering 3356 & 08.10.25  \\
	\hline
	\centering 08.10.25 & \centering 2 & Многоцелевая фильтрация текста с использованием регулярных выражений & \centering 10.06.2024 & \centering 2124 & 05.11.25  \\
	\hline
	\centering 22.10.25 & \centering 4 & & & &  \\
	\hline
	\centering 05.11.25 & \centering 5 & & & &  \\
	\hline
	\centering 19.11.25 & \centering 6 & & & &  \\
	\hline
	\centering 03.12.25 & \centering 7 & & & & \\
\hline

\hline
\end{tabular}

\begin{center}
\quad Выполнил(а) \underline{\hspace{1.5cm}Бых Даниил Максимович\hspace{1.5cm}}, № группы \underline{ P3109 }, оценка \underline{\hspace{2cm}}
\end{center}

\begin{tabularx}{\textwidth} { 
	| >{\raggedright\arraybackslash}X|} \hline
		\textbf{Прямая полная ссылка на источник или сокращённая ссылка} \\
		\url{https://cyberleninka.ru/article/n/mnogotselevaya-filtratsiya-teksta-s-ispolzovaniem-regulyarnyh-vyrazheniy/viewer}
		\smallskip\\
		\hline
		\textbf{Теги, ключевые слова или словосочетания}\\
		Многоцелевая фильтрация текста, паттерн, регулярные выражения, каскад.
		\smallskip\\
		\hline
		\textbf{Перечень фактов, упомянутых в статье}\\
		1) Фильтрация текста важна для автоматизированной обработки больших объемов динамичной информации. \\
		2) Фильтрация, основанная применении нескольких регулярных выражений для удаления избыточной информации, называется каскадной фильтрацией. \\
		3) Текстовый фильтр реализуется как алгоритм, который ищет и удаляет фрагменты текста по паттернам, которые описываются регулярными выражениями. \\
		4) Многоцелевая фильтрация эффективна при наличии множества разнообразных шаблонов.\\
		5) Существует проблема в требовании точного совпадения паттернов и логарифмическую зависимость скорости фильтрации от сложности шаблонов. \\
		\hline
		\textbf{Позитивные следствия и/или достоинства описанной в статье технологии}\\
		1) Позволяет гибко и последовательно отсеивать множество видов избыточной информации. \\
		2) Позволяет облегчить отладку и расширение фильтров без изменения существующих паттернов. \\
		3) Может использоваться для валидация данных, фильтрация пользовательских запросов, обработка текста в социальных сетях и программного кода. \\
		\hline
		\textbf{Негативные следствия и/или достоинства описанной в статье технологии}\\
		1) Требуется высококвалифицированный специалист для разработки и поддержки регулярных выражений. \\
		2) Строгое требование точного совпадения с паттерном усложняет охват всех вариаций текста. \\
		3) Скорость работы фильтра зависит логарифмически от количества и сложности правил. \\
		\hline
		\textbf{Ваши замечания, пожелания преподавателю или анекдот о программистах}\\
		Если у тебя есть проблема, и ты решил её решить с помощью регулярных выражений — значит у тебя теперь две проблемы.
		\bigskip\\
		\hline
		
\end{tabularx}


\end{document}
