%2025_Шаблон аннотации (.tex)
\documentclass[12pt]{article}
\pagestyle{empty}
\usepackage[utf8]{inputenc}
\usepackage[russian]{babel}
\usepackage[top=1cm,left=1cm,right=2cm, bottom=1cm]{geometry}
\usepackage{tabularx}% http://ctan.org/pkg/tabularx
%\usepackage{hyperref}
\begin{document}

\begin{center}
\quad Университет ИТМО, факультет программной инженерии и компьютерной техники \\
\quad Двухнедельная отчётная работа по «Информатике»: аннотация к статье\\
\end{center}

\begin{tabular}{
  |p{\dimexpr.1\linewidth-2\tabcolsep-1.3333\arrayrulewidth}% column 1
  |p{\dimexpr.1\linewidth-2\tabcolsep-1.3333\arrayrulewidth}% column 2
  |p{\dimexpr.47\linewidth-2\tabcolsep-1.3333\arrayrulewidth}% column 3
  |p{\dimexpr.13\linewidth-2\tabcolsep-1.3333\arrayrulewidth}% column 4
  |p{\dimexpr.1\linewidth-2\tabcolsep-1.3333\arrayrulewidth}% column 5
  |p{\dimexpr.1\linewidth-2\tabcolsep-1.3333\arrayrulewidth}|% column 6
  }
  \hline
  \centering\small{Дата прошедшей лекции} & \centering\small{Номер прошедшей лекции} & \centering\small{Название статьи/главы книги/видеолекции} & \centering\small{Дата публикации} & \centering\small{Размер статьи} & \centering\arraybackslash \small{Дата сдачи} \\ 
 \hline
 \centering 10.09.25 & \centering 1 & Использование изменения системы отсчета для улучшения результатов анализа с помощью закона бенфорда & \centering 31.03.2025 & \centering 1208 & 24.09.25  \\
  \hline

\hline
\end{tabular}

\begin{center}
\quad Выполнил(а) \underline{\hspace{1.5cm}Бых Даниил Максимович\hspace{1.5cm}}, № группы \underline{ P3109 }, оценка \underline{\hspace{2cm}}
\end{center}

\begin{tabularx}{\textwidth} { 
	| >{\raggedright\arraybackslash}X|} \hline
		\textbf{Прямая полная ссылка на источник или сокращённая ссылка} \\
		\url{https://cyberleninka.ru/article/n/ispolzovanie-izmeneniya-sistemy-otscheta-dlya-uluchsheniya-rezultatov-analiza-s-pomoschyu-zakona-benforda/viewer}
		\smallskip\\
		\hline
		\textbf{Теги, ключевые слова или словосочетания}\\
		Закон Бенфорда, анализ данных, системы счисления, статистика.
		\smallskip\\
		\hline
		\textbf{Перечень фактов, упомянутых в статье}\\
		1) Фрэнк Бенфорд вывел закон, по которому частота первых цифр чисел следует соотношению ${ P(n) = \log_b{(1 + \frac{1}{n})}}$, b - основание системы счисления, n - рассматриваемая цифра. \\
		2) Закон Бенфорда часто используют при проверке фальсификации выборов, для обнаружения мошеннических схем в финансовом секторе и в других областях, требующих обнаруживать аномалии на большой выборке данных. \\
		3) Некоторые действия над выборкой данных меняют распределение первых цифр так, что закон Бенфорда перестает действовать. \\
		4) Анализ становится явно точнее при увеличении системы счисления вплоть до тридцатишестеричной, а затем результат сильно варируется. \\
		5) На маленьких системах счисления аномалии в выборке становятся незаметными из-за небольшого числа значений цифр. \\
		6) Для эффективного анализа основание системы счисления должно быть минимум в 5 раз меньше числа анализируемых данных. \\
		\hline
		\textbf{Позитивные следствия и/или достоинства описанной в статье технологии}\\
		1) Если верно подобрать систему счисления, точность анализа возрастает в сравнении с исходной системой. \\
		2) По мере увеличения системы счисления любые аномалии в выборке данных становятся более выраженными. \\
		3) Аномалии, специально скрытые в исходной системе счисления, могут стать заметными при замене системы счисления. \\
		\hline
		\textbf{Негативные следствия и/или достоинства описанной в статье технологии}\\
		1) Так как анализ происходит не в привычной десятичной системе счисления, а какой-либо другой, обработка и подготовка данных может занимать больше времени. \\
		2) При приближении основания системы счисления к количеству данных точность сильно падает. \\
		3) В некоторых случаях метод может увеличивать точность только в случае дополнительного анализа второй, третьей и т.д. цифр. \\
		\hline
		\textbf{Ваши замечания, пожелания преподавателю или анекдот о программистах}\\
		Сколько нужно программистов, чтобы закрутить лампочку? Ни одного, это аппаратная проблема.
		\bigskip\\
		\hline
		
\end{tabularx}


\end{document}
