% !TeX program = lualatex

\documentclass{article}
 
\usepackage{mathtools}
\usepackage{array}
\usepackage{multirow}
\usepackage[russian]{babel}
\usepackage{titling}
\usepackage{adjustbox}
\usepackage{tikz}
\usepackage{tabularx}
\usepackage{multirow}
\usepackage{makecell}
\usepackage{tikz-inet}
\usepackage{graphicx}
\usetikzlibrary{matrix,shapes}
\usetikzlibrary {arrows.meta,graphs,graphdrawing}
\usetikzlibrary {circuits.logic.IEC}
\usetikzlibrary{positioning}
\usepackage{amssymb}
\usepackage{longtable}
\usepackage{karnaugh-map}
\usepackage{breqn}
\usepackage[pdf]{graphviz}
\usepackage[a4paper,left=2cm,right=2cm,top=2cm,bottom=1cm,footskip=.5cm]{geometry}
\usepackage{dot2texi}
\usepackage{tikz}

\usepackage{fontspec}
\setmainfont{CMU Serif}
\setsansfont{CMU Sans Serif}
\setmonofont{CMU Typewriter Text}

\newcommand{\tikzmark}[2]{\tikz[overlay,remember picture,baseline] 
\node [anchor=base] (#1) {$#2$};}

\newcommand*\circled[1]{\tikz[baseline=(char.base)]{
            \node[shape=circle,draw,inner sep=0pt] (char) {#1};}}

\newcommand*{\carry}[1][1]{\overset{#1}}
\newcolumntype{B}[1]{r*{#1}{@{\,}r}}

\usepackage{enumitem}
\makeatletter
\AddEnumerateCounter{\asbuk}{\russian@alph}{щ}
\makeatother

\setlength{\parindent}{0cm}
\setlength{\parskip}{1em}
\setlength{\fboxsep}{1pt}

\newcommand{\DrawVLine}[3][]{%
  \begin{tikzpicture}[overlay,remember picture]
    \draw[shorten <=0.3ex, #1] (#2.north) -- (#3.south);
  \end{tikzpicture}
}

\begin{document}

\begin{center}
    УНИВЕРСИТЕТ ИТМО \\
    Факультет программной инженерии и компьютерной техники \\
    Дисциплина «Дискретная математика»
    
    \vspace{5cm}

    \large
    \textbf{Курсовая работа} \\
    Часть 1 \\
    Вариант 9
\end{center}

\vspace{2cm}

\hfill\begin{minipage}{0.35\linewidth}
Студент \\
Бых Даниил Максимович \\
P3109 \\

Преподаватель \\
Поляков Владимир Иванович
\end{minipage}

\vfill

\begin{center}
    Санкт-Петербург, 2025 г.
\end{center}

\thispagestyle{empty}
\newpage

Функция $f(x_1, x_2, x_3, x_4, x_5)$ принимает значение 1 при $3 \textless (x_4 x_5 + x_1 x_2 x_3) \textless 8$ и неопределенное значение при $(x_1 x_2 x_3) = 1$.

\section*{Таблица истинности}
\begin{center}\begin{tabular}{|c|ccccc|c*{4}{|c}|}
    \hline
    № & $x_1$ & $x_2$ & $x_3$ & $x_4$ & $x_5$  & $ x_4  x_5 $ & $ x_1  x_2  x_3 $ & $ x_1  x_2  x_3 $& $f$ \\ \hline
    0 & 0 & 0 & 0 & 0 & 0 & 0 & 0 & 0 & 0 \\ \hline
    1 & 0 & 0 & 0 & 0 & 1 & 1 & 0 & 0 & 0 \\ \hline
    2 & 0 & 0 & 0 & 1 & 0 & 2 & 0 & 0 & 0 \\ \hline
    3 & 0 & 0 & 0 & 1 & 1 & 3 & 0 & 0 & 0 \\ \hline
    4 & 0 & 0 & 1 & 0 & 0 & 0 & 1 & 1 & d \\ \hline
    5 & 0 & 0 & 1 & 0 & 1 & 1 & 1 & 1 & d \\ \hline
    6 & 0 & 0 & 1 & 1 & 0 & 2 & 1 & 1 & d \\ \hline
    7 & 0 & 0 & 1 & 1 & 1 & 3 & 1 & 1 & d \\ \hline
    8 & 0 & 1 & 0 & 0 & 0 & 0 & 2 & 2 & 0 \\ \hline
    9 & 0 & 1 & 0 & 0 & 1 & 1 & 2 & 2 & 0 \\ \hline
    10 & 0 & 1 & 0 & 1 & 0 & 2 & 2 & 2 & 1 \\ \hline
    11 & 0 & 1 & 0 & 1 & 1 & 3 & 2 & 2 & 1 \\ \hline
    12 & 0 & 1 & 1 & 0 & 0 & 0 & 3 & 3 & 0 \\ \hline
    13 & 0 & 1 & 1 & 0 & 1 & 1 & 3 & 3 & 1 \\ \hline
    14 & 0 & 1 & 1 & 1 & 0 & 2 & 3 & 3 & 1 \\ \hline
    15 & 0 & 1 & 1 & 1 & 1 & 3 & 3 & 3 & 1 \\ \hline
    16 & 1 & 0 & 0 & 0 & 0 & 0 & 4 & 4 & 1 \\ \hline
    17 & 1 & 0 & 0 & 0 & 1 & 1 & 4 & 4 & 1 \\ \hline
    18 & 1 & 0 & 0 & 1 & 0 & 2 & 4 & 4 & 1 \\ \hline
    19 & 1 & 0 & 0 & 1 & 1 & 3 & 4 & 4 & 1 \\ \hline
    20 & 1 & 0 & 1 & 0 & 0 & 0 & 5 & 5 & 1 \\ \hline
    21 & 1 & 0 & 1 & 0 & 1 & 1 & 5 & 5 & 1 \\ \hline
    22 & 1 & 0 & 1 & 1 & 0 & 2 & 5 & 5 & 1 \\ \hline
    23 & 1 & 0 & 1 & 1 & 1 & 3 & 5 & 5 & 0 \\ \hline
    24 & 1 & 1 & 0 & 0 & 0 & 0 & 6 & 6 & 1 \\ \hline
    25 & 1 & 1 & 0 & 0 & 1 & 1 & 6 & 6 & 1 \\ \hline
    26 & 1 & 1 & 0 & 1 & 0 & 2 & 6 & 6 & 0 \\ \hline
    27 & 1 & 1 & 0 & 1 & 1 & 3 & 6 & 6 & 0 \\ \hline
    28 & 1 & 1 & 1 & 0 & 0 & 0 & 7 & 7 & 1 \\ \hline
    29 & 1 & 1 & 1 & 0 & 1 & 1 & 7 & 7 & 0 \\ \hline
    30 & 1 & 1 & 1 & 1 & 0 & 2 & 7 & 7 & 0 \\ \hline
    31 & 1 & 1 & 1 & 1 & 1 & 3 & 7 & 7 & 0 \\ \hline
\end{tabular}\end{center}
\section*{Аналитический вид}
\subsection*{Каноническая ДНФ:}
\begin{align*}
f =\: &\overline{x_{1}} \, x_{2} \, \overline{x_{3}} \, x_{4} \, \overline{x_{5}}\lor \overline{x_{1}} \, x_{2} \, \overline{x_{3}} \, x_{4} \, x_{5}\lor \overline{x_{1}} \, x_{2} \, x_{3} \, \overline{x_{4}} \, x_{5}\lor \overline{x_{1}} \, x_{2} \, x_{3} \, x_{4} \, \overline{x_{5}}\lor \overline{x_{1}} \, x_{2} \, x_{3} \, x_{4} \, x_{5}\lor x_{1} \, \overline{x_{2}} \, \overline{x_{3}} \, \overline{x_{4}} \, \overline{x_{5}}\lor \\ \lor\: &x_{1} \, \overline{x_{2}} \, \overline{x_{3}} \, \overline{x_{4}} \, x_{5}\lor x_{1} \, \overline{x_{2}} \, \overline{x_{3}} \, x_{4} \, \overline{x_{5}}\lor x_{1} \, \overline{x_{2}} \, \overline{x_{3}} \, x_{4} \, x_{5}\lor x_{1} \, \overline{x_{2}} \, x_{3} \, \overline{x_{4}} \, \overline{x_{5}}\lor x_{1} \, \overline{x_{2}} \, x_{3} \, \overline{x_{4}} \, x_{5}\lor x_{1} \, \overline{x_{2}} \, x_{3} \, x_{4} \, \overline{x_{5}}\lor \\ \lor\: &x_{1} \, x_{2} \, \overline{x_{3}} \, \overline{x_{4}} \, \overline{x_{5}}\lor x_{1} \, x_{2} \, \overline{x_{3}} \, \overline{x_{4}} \, x_{5}\lor x_{1} \, x_{2} \, x_{3} \, \overline{x_{4}} \, \overline{x_{5}}\end{align*}
\subsection*{Каноническая КНФ:}
\begin{align*}
f =\: &\left(x_{1} \lor x_{2} \lor x_{3} \lor x_{4} \lor x_{5}\right)\left(x_{1} \lor x_{2} \lor x_{3} \lor x_{4} \lor \overline{x_{5}}\right)\left(x_{1} \lor x_{2} \lor x_{3} \lor \overline{x_{4}} \lor x_{5}\right)\left(x_{1} \lor x_{2} \lor x_{3} \lor \overline{x_{4}} \lor \overline{x_{5}}\right)\\&\left(x_{1} \lor \overline{x_{2}} \lor x_{3} \lor x_{4} \lor x_{5}\right)\left(x_{1} \lor \overline{x_{2}} \lor x_{3} \lor x_{4} \lor \overline{x_{5}}\right)\left(x_{1} \lor \overline{x_{2}} \lor \overline{x_{3}} \lor x_{4} \lor x_{5}\right)\left(\overline{x_{1}} \lor x_{2} \lor \overline{x_{3}} \lor \overline{x_{4}} \lor \overline{x_{5}}\right)\\&\left(\overline{x_{1}} \lor \overline{x_{2}} \lor x_{3} \lor \overline{x_{4}} \lor x_{5}\right)\left(\overline{x_{1}} \lor \overline{x_{2}} \lor x_{3} \lor \overline{x_{4}} \lor \overline{x_{5}}\right)\left(\overline{x_{1}} \lor \overline{x_{2}} \lor \overline{x_{3}} \lor x_{4} \lor \overline{x_{5}}\right)\left(\overline{x_{1}} \lor \overline{x_{2}} \lor \overline{x_{3}} \lor \overline{x_{4}} \lor x_{5}\right)\\&\left(\overline{x_{1}} \lor \overline{x_{2}} \lor \overline{x_{3}} \lor \overline{x_{4}} \lor \overline{x_{5}}\right)\end{align*}
\section*{Минимизация булевой функции методом Квайна--Мак-Класки}
\subsection*{Кубы различной размерности и простые импликанты}
\begin{center}
\begin{tabular}[t]{|lcc|}
\hline \multicolumn{3}{|c|}{$K^0(f)$}\\ \hline
$m_{16}$ & 10000& \checkmark \\$m_{4}$ & 00100& \checkmark \\\hline
$m_{10}$ & 01010& \checkmark \\$m_{17}$ & 10001& \checkmark \\$m_{18}$ & 10010& \checkmark \\$m_{20}$ & 10100& \checkmark \\$m_{24}$ & 11000& \checkmark \\$m_{5}$ & 00101& \checkmark \\$m_{6}$ & 00110& \checkmark \\\hline
$m_{11}$ & 01011& \checkmark \\$m_{13}$ & 01101& \checkmark \\$m_{14}$ & 01110& \checkmark \\$m_{19}$ & 10011& \checkmark \\$m_{21}$ & 10101& \checkmark \\$m_{22}$ & 10110& \checkmark \\$m_{25}$ & 11001& \checkmark \\$m_{28}$ & 11100& \checkmark \\$m_{7}$ & 00111& \checkmark \\\hline
$m_{15}$ & 01111& \checkmark \\\hline
\end{tabular}
\begin{tabular}[t]{|lcc|}
\hline \multicolumn{3}{|c|}{$K^1(f)$}\\ \hline
$m_{4}\mbox{-}m_{5}$ & 0010X& \checkmark \\$m_{4}\mbox{-}m_{6}$ & 001X0& \checkmark \\$m_{16}\mbox{-}m_{17}$ & 1000X& \checkmark \\$m_{16}\mbox{-}m_{18}$ & 100X0& \checkmark \\$m_{16}\mbox{-}m_{20}$ & 10X00& \checkmark \\$m_{16}\mbox{-}m_{24}$ & 1X000& \checkmark \\$m_{4}\mbox{-}m_{20}$ & X0100& \checkmark \\\hline
$m_{6}\mbox{-}m_{7}$ & 0011X& \checkmark \\$m_{5}\mbox{-}m_{7}$ & 001X1& \checkmark \\$m_{10}\mbox{-}m_{11}$ & 0101X& \checkmark \\$m_{10}\mbox{-}m_{14}$ & 01X10& \checkmark \\$m_{5}\mbox{-}m_{13}$ & 0X101& \checkmark \\$m_{6}\mbox{-}m_{14}$ & 0X110& \checkmark \\$m_{18}\mbox{-}m_{19}$ & 1001X& \checkmark \\$m_{17}\mbox{-}m_{19}$ & 100X1& \checkmark \\$m_{20}\mbox{-}m_{21}$ & 1010X& \checkmark \\$m_{20}\mbox{-}m_{22}$ & 101X0& \checkmark \\$m_{17}\mbox{-}m_{21}$ & 10X01& \checkmark \\$m_{18}\mbox{-}m_{22}$ & 10X10& \checkmark \\$m_{24}\mbox{-}m_{25}$ & 1100X& \checkmark \\$m_{24}\mbox{-}m_{28}$ & 11X00& \checkmark \\$m_{17}\mbox{-}m_{25}$ & 1X001& \checkmark \\$m_{20}\mbox{-}m_{28}$ & 1X100& \checkmark \\$m_{5}\mbox{-}m_{21}$ & X0101& \checkmark \\$m_{6}\mbox{-}m_{22}$ & X0110& \checkmark \\\hline
$m_{14}\mbox{-}m_{15}$ & 0111X& \checkmark \\$m_{13}\mbox{-}m_{15}$ & 011X1& \checkmark \\$m_{11}\mbox{-}m_{15}$ & 01X11& \checkmark \\$m_{7}\mbox{-}m_{15}$ & 0X111& \checkmark \\\hline
\end{tabular}
\begin{tabular}[t]{|lcc|}
\hline \multicolumn{3}{|c|}{$K^2(f)$}\\ \hline
$m_{4}\mbox{-}m_{5}\mbox{-}m_{6}\mbox{-}m_{7}$ & 001XX& \\$m_{16}\mbox{-}m_{17}\mbox{-}m_{18}\mbox{-}m_{19}$ & 100XX& \\$m_{16}\mbox{-}m_{17}\mbox{-}m_{20}\mbox{-}m_{21}$ & 10X0X& \\$m_{16}\mbox{-}m_{18}\mbox{-}m_{20}\mbox{-}m_{22}$ & 10XX0& \\$m_{16}\mbox{-}m_{17}\mbox{-}m_{24}\mbox{-}m_{25}$ & 1X00X& \\$m_{16}\mbox{-}m_{20}\mbox{-}m_{24}\mbox{-}m_{28}$ & 1XX00& \\$m_{4}\mbox{-}m_{5}\mbox{-}m_{20}\mbox{-}m_{21}$ & X010X& \\$m_{4}\mbox{-}m_{6}\mbox{-}m_{20}\mbox{-}m_{22}$ & X01X0& \\\hline
$m_{10}\mbox{-}m_{11}\mbox{-}m_{14}\mbox{-}m_{15}$ & 01X1X& \\$m_{6}\mbox{-}m_{7}\mbox{-}m_{14}\mbox{-}m_{15}$ & 0X11X& \\$m_{5}\mbox{-}m_{7}\mbox{-}m_{13}\mbox{-}m_{15}$ & 0X1X1& \\\hline
\end{tabular}
\begin{tabular}[t]{|c|}
\hline $Z(f)$ \\ \hline
001XX\\
100XX\\
10X0X\\
10XX0\\
1X00X\\
1XX00\\
X010X\\
X01X0\\
01X1X\\
0X11X\\
0X1X1\\
\hline \end{tabular}
\end{center}
\subsection*{Таблица импликант}
Необходимо вычеркнуть строки, соответствующие существенным импликантам (те, что покрывают вершины, которые не покрыты другими импликантами), столбцы, которые соответствуют вершинам, покрываемым существенными импликантами. Кроме того, вычеркнем импликанты, которые не покрывают ни одной вершины.
\newpage
\begin{flushleft}\begin{tabular}{|c|c|r*{15}{|c}|}
    \hline \multicolumn{2}{|c|}{\multirow{7}{*}{Простые импликанты}} & \multicolumn{15}{c|}{0-кубы} \\ \cline{3-17}
    \multicolumn{2}{|c|}{} & \makecell{\tikzmark{start_0}{0}} & \makecell{\tikzmark{start_1}{0}} & \makecell{\tikzmark{start_2}{0}} & \makecell{\tikzmark{start_3}{0}} & \makecell{\tikzmark{start_4}{0}} & \makecell{\tikzmark{start_5}{1}} & \makecell{\tikzmark{start_6}{1}} & \makecell{\tikzmark{start_7}{1}} & \makecell{\tikzmark{start_8}{1}} & \makecell{\tikzmark{start_9}{1}} & \makecell{\tikzmark{start_10}{1}} & \makecell{\tikzmark{start_11}{1}} & \makecell{\tikzmark{start_12}{1}} & \makecell{\tikzmark{start_13}{1}} & \makecell{\tikzmark{start_14}{1}}\\
    \multicolumn{2}{|c|}{} & \makecell{1} & \makecell{1} & \makecell{1} & \makecell{1} & \makecell{1} & \makecell{0} & \makecell{0} & \makecell{0} & \makecell{0} & \makecell{0} & \makecell{0} & \makecell{0} & \makecell{1} & \makecell{1} & \makecell{1}\\
    \multicolumn{2}{|c|}{} & \makecell{0} & \makecell{0} & \makecell{1} & \makecell{1} & \makecell{1} & \makecell{0} & \makecell{0} & \makecell{0} & \makecell{0} & \makecell{1} & \makecell{1} & \makecell{1} & \makecell{0} & \makecell{0} & \makecell{1}\\
    \multicolumn{2}{|c|}{} & \makecell{1} & \makecell{1} & \makecell{0} & \makecell{1} & \makecell{1} & \makecell{0} & \makecell{0} & \makecell{1} & \makecell{1} & \makecell{0} & \makecell{0} & \makecell{1} & \makecell{0} & \makecell{0} & \makecell{0}\\
    \multicolumn{2}{|c|}{} & \makecell{0} & \makecell{1} & \makecell{1} & \makecell{0} & \makecell{1} & \makecell{0} & \makecell{1} & \makecell{0} & \makecell{1} & \makecell{0} & \makecell{1} & \makecell{0} & \makecell{0} & \makecell{1} & \makecell{0}\\
    \cline{3-17}
    \multicolumn{2}{|c|}{} & \makecell{10} & \makecell{11} & \makecell{13} & \makecell{14} & \makecell{15} & \makecell{16} & \makecell{17} & \makecell{18} & \makecell{19} & \makecell{20} & \makecell{21} & \makecell{22} & \makecell{24} & \makecell{25} & \makecell{28}\\ \hline
    & 001XX&\makecell{ }&\makecell{ }&\makecell{ }&\makecell{ }&\makecell{ }&\makecell{ }&\makecell{ }&\makecell{ }&\makecell{ }&\makecell{ }&\makecell{ }&\makecell{ }&\makecell{ }&\makecell{ }&\makecell{ }\\ [-1.6ex] \hline\noalign{\vspace{\dimexpr 1.6ex-\doublerulesep}} \hline
    & 100XX&\makecell{ }&\makecell{ }&\makecell{ }&\makecell{ }&\makecell{ }&\makecell{X}&\makecell{X}&\makecell{X}&\makecell{X}&\makecell{ }&\makecell{ }&\makecell{ }&\makecell{ }&\makecell{ }&\makecell{ }\\ [-1.6ex] \hline\noalign{\vspace{\dimexpr 1.6ex-\doublerulesep}} \hline
    A & 10X0X&\makecell{ }&\makecell{ }&\makecell{ }&\makecell{ }&\makecell{ }&\makecell{X}&\makecell{X}&\makecell{ }&\makecell{ }&\makecell{X}&\makecell{X}&\makecell{ }&\makecell{ }&\makecell{ }&\makecell{ }\\ \hline
    B & 10XX0&\makecell{ }&\makecell{ }&\makecell{ }&\makecell{ }&\makecell{ }&\makecell{X}&\makecell{ }&\makecell{X}&\makecell{ }&\makecell{X}&\makecell{ }&\makecell{X}&\makecell{ }&\makecell{ }&\makecell{ }\\ \hline
    & 1X00X&\makecell{ }&\makecell{ }&\makecell{ }&\makecell{ }&\makecell{ }&\makecell{X}&\makecell{X}&\makecell{ }&\makecell{ }&\makecell{ }&\makecell{ }&\makecell{ }&\makecell{X}&\makecell{X}&\makecell{ }\\ [-1.6ex] \hline\noalign{\vspace{\dimexpr 1.6ex-\doublerulesep}} \hline
    & 1XX00&\makecell{ }&\makecell{ }&\makecell{ }&\makecell{ }&\makecell{ }&\makecell{X}&\makecell{ }&\makecell{ }&\makecell{ }&\makecell{X}&\makecell{ }&\makecell{ }&\makecell{X}&\makecell{ }&\makecell{X}\\ [-1.6ex] \hline\noalign{\vspace{\dimexpr 1.6ex-\doublerulesep}} \hline
    C & X010X&\makecell{ }&\makecell{ }&\makecell{ }&\makecell{ }&\makecell{ }&\makecell{ }&\makecell{ }&\makecell{ }&\makecell{ }&\makecell{X}&\makecell{X}&\makecell{ }&\makecell{ }&\makecell{ }&\makecell{ }\\ \hline
    D & X01X0&\makecell{ }&\makecell{ }&\makecell{ }&\makecell{ }&\makecell{ }&\makecell{ }&\makecell{ }&\makecell{ }&\makecell{ }&\makecell{X}&\makecell{ }&\makecell{X}&\makecell{ }&\makecell{ }&\makecell{ }\\ \hline
    & 01X1X&\makecell{X}&\makecell{X}&\makecell{ }&\makecell{X}&\makecell{X}&\makecell{ }&\makecell{ }&\makecell{ }&\makecell{ }&\makecell{ }&\makecell{ }&\makecell{ }&\makecell{ }&\makecell{ }&\makecell{ }\\ [-1.6ex] \hline\noalign{\vspace{\dimexpr 1.6ex-\doublerulesep}} \hline
    & 0X11X&\makecell{ }&\makecell{ }&\makecell{ }&\makecell{X}&\makecell{X}&\makecell{ }&\makecell{ }&\makecell{ }&\makecell{ }&\makecell{ }&\makecell{ }&\makecell{ }&\makecell{ }&\makecell{ }&\makecell{ }\\ [-1.6ex] \hline\noalign{\vspace{\dimexpr 1.6ex-\doublerulesep}} \hline
    & 0X1X1&\makecell{\tikzmark{end_0}{ }}&\makecell{\tikzmark{end_1}{ }}&\makecell{\tikzmark{end_2}{X}}&\makecell{\tikzmark{end_3}{ }}&\makecell{\tikzmark{end_4}{X}}&\makecell{\tikzmark{end_5}{ }}&\makecell{\tikzmark{end_6}{ }}&\makecell{\tikzmark{end_7}{ }}&\makecell{\tikzmark{end_8}{ }}&\makecell{\tikzmark{end_9}{ }}&\makecell{\tikzmark{end_10}{ }}&\makecell{\tikzmark{end_11}{ }}&\makecell{\tikzmark{end_12}{ }}&\makecell{\tikzmark{end_13}{ }}&\makecell{\tikzmark{end_14}{ }}\\ [-1.6ex] \hline\noalign{\vspace{\dimexpr 1.6ex-\doublerulesep}} \hline
\end{tabular}\end{flushleft}
\DrawVLine[black]{start_0}{end_0}
\DrawVLine[black]{start_1}{end_1}
\DrawVLine[black]{start_2}{end_2}
\DrawVLine[black]{start_3}{end_3}
\DrawVLine[black]{start_4}{end_4}
\DrawVLine[black]{start_5}{end_5}
\DrawVLine[black]{start_6}{end_6}
\DrawVLine[black]{start_7}{end_7}
\DrawVLine[black]{start_8}{end_8}
\DrawVLine[black]{start_9}{end_9}
\DrawVLine[black]{start_12}{end_12}
\DrawVLine[black]{start_13}{end_13}
\DrawVLine[black]{start_14}{end_14}

Ядро покрытия:
\[T = \begin{Bmatrix}0X1X1\\01X1X\\100XX\\1X00X\\1XX00\end{Bmatrix}\]

В результате получаем упрощенную импликантную таблицу:
\begin{flushleft}\begin{tabular}{|c|c|r*{2}{|c}|}
    \hline \multicolumn{2}{|c|}{\multirow{7}{*}{Простые импликанты}} & \multicolumn{2}{c|}{0-кубы} \\ \cline{3-4}
    \multicolumn{2}{|c|}{} & \makecell{\tikzmark{start_100}{1}} & \makecell{\tikzmark{start_101}{1}}\\
    \multicolumn{2}{|c|}{} & \makecell{0} & \makecell{0}\\
    \multicolumn{2}{|c|}{} & \makecell{1} & \makecell{1}\\
    \multicolumn{2}{|c|}{} & \makecell{0} & \makecell{1}\\
    \multicolumn{2}{|c|}{} & \makecell{1} & \makecell{0}\\
    \cline{3-4}
    \multicolumn{2}{|c|}{} & \makecell{21} & \makecell{22}\\ \hline
    A & 10X0X&\makecell{X}&\makecell{ }\\ \hline
    B & 10XX0&\makecell{ }&\makecell{X}\\ \hline
    C & X010X&\makecell{X}&\makecell{ }\\ \hline
    D & X01X0&\makecell{\tikzmark{end_100}{ }}&\makecell{\tikzmark{end_101}{X}}\\ \hline
\end{tabular}\end{flushleft}

Метод Петрика:


Сначала запишем булево выражение, определяющее условие покрытия всех вершин:

$Y = \left(A \lor C\right) \, \left(B \lor D\right)$

Затем приведем выражение в ДНФ:

$Y = A \, B \lor A \, D \lor B \, C \lor C \, D$

Тогда возможны следующие покрытия:
\begin{center}\begin{tabular}{cccc}
$\begin{array}{c}
C_{1} = \begin{Bmatrix} T\\ A\\ B\end{Bmatrix} = \begin{Bmatrix}0X1X1\\01X1X\\100XX\\1X00X\\1XX00\\ 10X0X\\ 10XX0\end{Bmatrix} \\ \\
S^a_{1} = 21 \\
S^b_{1} = 28 \\ \phantom{0}
\end{array}$
 & $\begin{array}{c}
C_{2} = \begin{Bmatrix} T\\ A\\ D\end{Bmatrix} = \begin{Bmatrix}0X1X1\\01X1X\\100XX\\1X00X\\1XX00\\ 10X0X\\ X01X0\end{Bmatrix} \\ \\
S^a_{2} = 21 \\
S^b_{2} = 28 \\ \phantom{0}
\end{array}$
 & $\begin{array}{c}
C_{3} = \begin{Bmatrix} T\\ B\\ C\end{Bmatrix} = \begin{Bmatrix}0X1X1\\01X1X\\100XX\\1X00X\\1XX00\\ 10XX0\\ X010X\end{Bmatrix} \\ \\
S^a_{3} = 21 \\
S^b_{3} = 28 \\ \phantom{0}
\end{array}$
\\
$\begin{array}{c}
C_{4} = \begin{Bmatrix} T\\ C\\ D\end{Bmatrix} = \begin{Bmatrix}0X1X1\\01X1X\\100XX\\1X00X\\1XX00\\ X010X\\ X01X0\end{Bmatrix} \\ \\
S^a_{4} = 21 \\
S^b_{4} = 28 \\ \phantom{0}
\end{array}$
\\
\end{tabular}\end{center}

Рассмотрим минимальное покрытие:
\[\begin{array}{c}
C_{\text{min}} = \begin{Bmatrix}0X1X1\\01X1X\\100XX\\1X00X\\1XX00\\10X0X\\10XX0\end{Bmatrix} \\ \\
S^a = 21 \\
S^b = 28
\end{array}\]

Данному покрытию соответствует МДНФ:
\[f = \overline{x_{1}} \, x_{3} \, x_{5} \lor \overline{x_{1}} \, x_{2} \, x_{4} \lor x_{1} \, \overline{x_{2}} \, \overline{x_{3}} \lor x_{1} \, \overline{x_{3}} \, \overline{x_{4}} \lor x_{1} \, \overline{x_{4}} \, \overline{x_{5}} \lor x_{1} \, \overline{x_{2}} \, \overline{x_{4}} \lor x_{1} \, \overline{x_{2}} \, \overline{x_{5}}\]
\section*{Минимизация булевой функции на картах Карно}
\subsection*{Определение МДНФ}
\begin{minipage}{0.7\textwidth}
\begin{karnaugh-map}[4][4][2][$x_4 x_5$][$x_2 x_3$][$x_1$]
    \minterms{10, 11, 13, 14, 15, 16, 17, 18, 19, 20, 21, 22, 24, 25, 28}
    \terms{4, 5, 6, 7}{d}
    \implicant{5}{15}[0]
    \implicant{15}{10}[0]
    \implicant{0}{2}[1]
    \implicantedge{0}{1}{8}{9}[1]
    \implicant{0}{8}[1]
    \implicant{0}{5}[1]
    \implicantedge{0}{4}{2}{6}[1]
\end{karnaugh-map}
\end{minipage}
\begin{minipage}{0.3\textwidth - 5pt}\vfill
\[\begin{array}{c}
C_{\text{min}} = \begin{Bmatrix}0X1X1\\01X1X\\100XX\\1X00X\\1XX00\\10X0X\\10XX0\end{Bmatrix} \\ \\
S^a = 21 \\
S^b = 28
\end{array}\]
\vfill\end{minipage}
\[f = \overline{x_{1}} \, x_{3} \, x_{5} \lor \overline{x_{1}} \, x_{2} \, x_{4} \lor x_{1} \, \overline{x_{2}} \, \overline{x_{3}} \lor x_{1} \, \overline{x_{3}} \, \overline{x_{4}} \lor x_{1} \, \overline{x_{4}} \, \overline{x_{5}} \lor x_{1} \, \overline{x_{2}} \, \overline{x_{4}} \lor x_{1} \, \overline{x_{2}} \, \overline{x_{5}}\]
\subsection*{Определение МКНФ}
\begin{minipage}{0.7\textwidth}
\begin{karnaugh-map}[4][4][2][$x_4 x_5$][$x_2 x_3$][$x_1$]
    \maxterms{0, 1, 2, 3, 8, 9, 12, 23, 26, 27, 29, 30, 31}
    \terms{4, 5, 6, 7}{d}
    \implicant{0}{6}[0]
    \implicantedge{0}{1}{8}{9}[0]
    \implicant{0}{8}[0]
    \implicant{15}{10}[1]
    \implicant{13}{15}[1]
    \implicant{7}{7}[0, 1]
\end{karnaugh-map}
\end{minipage}
\begin{minipage}{0.3\textwidth - 5pt}\vfill
\[\begin{array}{c}
C_{\text{min}} = \begin{Bmatrix}00XXX\\0X00X\\0XX00\\11X1X\\111X1\\X0111\end{Bmatrix} \\ \\
S^a = 19 \\
S^b = 25
\end{array}\]
\vfill\end{minipage}
\[f = \left(x_{1} \lor x_{2}\right) \, \left(x_{1} \lor x_{3} \lor x_{4}\right) \, \left(x_{1} \lor x_{4} \lor x_{5}\right) \, \left(\overline{x_{1}} \lor \overline{x_{2}} \lor \overline{x_{4}}\right) \, \left(\overline{x_{1}} \lor \overline{x_{2}} \lor \overline{x_{3}} \lor \overline{x_{5}}\right) \, \left(x_{2} \lor \overline{x_{3}} \lor \overline{x_{4}} \lor \overline{x_{5}}\right)\]
\section*{Преобразование минимальных форм булевой функции}
\subsection*{Факторизация и декомпозиция МДНФ}
\begin{flalign*}\def\arraystretch{1.5}\begin{array}{lll}
f = \overline{x_{1}} \, x_{3} \, x_{5} \lor \overline{x_{1}} \, x_{2} \, x_{4} \lor x_{1} \, \overline{x_{2}} \, \overline{x_{3}} \lor x_{1} \, \overline{x_{3}} \, \overline{x_{4}} \lor x_{1} \, \overline{x_{4}} \, \overline{x_{5}} \lor x_{1} \, \overline{x_{2}} \, \overline{x_{4}} \lor x_{1} \, \overline{x_{2}} \, \overline{x_{5}} & S_Q = 28 & \tau = 2 \\
f = \overline{x_{1}} \, \left(x_{3} \, x_{5} \lor x_{2} \, x_{4}\right) \lor x_{1} \, \left(\overline{x_{2}} \lor \overline{x_{4}}\right) \, \left(\overline{x_{3}} \lor \overline{x_{5}}\right) \lor x_{1} \, \overline{x_{2}} \, \overline{x_{4}} & S_Q = 21 & \tau = 4 \\
\varphi = \left(\overline{x_{3}} \lor \overline{x_{5}}\right) \, \left(\overline{x_{2}} \lor \overline{x_{4}}\right) \\
\overline{\varphi} = x_{3} \, x_{5} \lor x_{2} \, x_{4} \\
f = \overline{x_{1}} \, \overline{\varphi} \lor \varphi \, x_{1} \lor x_{1} \, \overline{x_{2}} \, \overline{x_{4}} & S_Q = 17 & \tau = 5 \\
\end{array}&&\end{flalign*}
\subsection*{Факторизация и декомпозиция МКНФ}
\begin{flalign*}\def\arraystretch{1.5}\begin{array}{lll}
f = \left(x_{1} \lor x_{2}\right) \, \left(x_{1} \lor x_{3} \lor x_{4}\right) \, \left(x_{1} \lor x_{4} \lor x_{5}\right) \, \left(\overline{x_{1}} \lor \overline{x_{2}} \lor \overline{x_{4}}\right) \, \left(\overline{x_{1}} \lor \overline{x_{2}} \lor \overline{x_{3}} \lor \overline{x_{5}}\right) \, \left(x_{2} \lor \overline{x_{3}} \lor \overline{x_{4}} \lor \overline{x_{5}}\right) & S_Q = 25 & \tau = 2 \\
f = \left(x_{1} \lor x_{2} \, \left(x_{4} \lor x_{3} \, x_{5}\right)\right) \, \left(\overline{x_{1}} \lor \overline{x_{2}} \lor \overline{x_{4}} \, \left(\overline{x_{3}} \lor \overline{x_{5}}\right)\right) \, \left(x_{2} \lor \overline{x_{3}} \lor \overline{x_{4}} \lor \overline{x_{5}}\right) & S_Q = 22 & \tau = 5 \\
\varphi = x_{2} \, \left(x_{4} \lor x_{3} \, x_{5}\right) \\
\overline{\varphi} = \overline{x_{2}} \lor \overline{x_{4}} \, \left(\overline{x_{3}} \lor \overline{x_{5}}\right) \\
f = \left(x_{1} \lor \varphi\right) \, \left(\overline{\varphi} \lor \overline{x_{1}}\right) \, \left(x_{2} \lor \overline{x_{3}} \lor \overline{x_{4}} \lor \overline{x_{5}}\right) & S_Q = 18 & \tau = 6 \\
\end{array}&&\end{flalign*}
\section*{Синтез комбинационных схем}
Будем анализировать схемы на следующих наборах аргументов:
\begin{align*}
    f([x_1 = 0, x_2 = 0, x_3 = 0, x_4 = 0, x_5 = 0]) &= 0 \\
    f([x_1 = 0, x_2 = 0, x_3 = 0, x_4 = 0, x_5 = 1]) &= 0 \\
    f([x_1 = 0, x_2 = 1, x_3 = 0, x_4 = 1, x_5 = 0]) &= 1 \\
    f([x_1 = 0, x_2 = 1, x_3 = 0, x_4 = 1, x_5 = 1]) &= 1 \\
\end{align*}
\subsection*{Булев базис}
Схема по упрощенной МДНФ:
\[f = \overline{x_{1}} \, \overline{\varphi} \lor \varphi \, x_{1} \lor x_{1} \, \overline{x_{2}} \, \overline{x_{4}}\quad(S_Q = 17, \tau = 5)\]
\[\varphi = \left(\overline{x_{3}} \lor \overline{x_{5}}\right) \, \left(\overline{x_{2}} \lor \overline{x_{4}}\right)\]
\begin{center}\begin{tikzpicture}[circuit logic IEC]
\node at (0,0) (n0) {$f$};
\node[and gate,inputs={nn}] at (-8,0.20000005) (n1) {};
\node[or gate,inputs={nn}] at (-9.5,-0.3499999) (n2) {};
\node at (-11,-0.5166666) (n3) {$\overline{x_4}$};
\draw (n2.input 2) -- ++(left:2mm) |- (n3.east) node[at end, above, xshift=2.0mm, yshift=-2pt]{\tiny\texttt{1100}};
\node at (-11,-0.18333325) (n4) {$\overline{x_2}$};
\draw (n2.input 1) -- ++(left:2mm) |- (n4.east) node[at end, above, xshift=2.0mm, yshift=-2pt]{\tiny\texttt{1100}};
\draw (n1.input 2) -- ++(left:2mm) |- (n2.output) node[at end, above, xshift=2.0mm, yshift=-2pt]{\tiny\texttt{1100}};
\node[or gate,inputs={nn}] at (-9.5,0.7500001) (n5) {};
\node at (-11,0.58333343) (n6) {$\overline{x_5}$};
\draw (n5.input 2) -- ++(left:2mm) |- (n6.east) node[at end, above, xshift=2.0mm, yshift=-2pt]{\tiny\texttt{1010}};
\node at (-11,0.91666675) (n7) {$\overline{x_3}$};
\draw (n5.input 1) -- ++(left:2mm) |- (n7.east) node[at end, above, xshift=2.0mm, yshift=-2pt]{\tiny\texttt{1111}};
\draw (n1.input 1) -- ++(left:2mm) |- (n5.output) node[at end, above, xshift=2.0mm, yshift=-2pt]{\tiny\texttt{1111}};
\node[not gate] at (-6.5,-1) (not) {};
\node[or gate,inputs={nnn}] at (-1,0) (n8) {};
\node[and gate,inputs={nnn}] at (-2.5,-1.2666667) (n9) {};
\node at (-4,-1.6) (n10) {$\overline{x_4}$};
\draw (n9.input 3) -- ++(left:2mm) |- (n10.east) node[at end, above, xshift=2.0mm, yshift=-2pt]{\tiny\texttt{1100}};
\node at (-4,-1.2666667) (n11) {$\overline{x_2}$};
\draw (n9.input 2) -- ++(left:3.5mm) |- (n11.east) node[at end, above, xshift=2.0mm, yshift=-2pt]{\tiny\texttt{1100}};
\node at (-4,-0.9333333) (n12) {$x_1$};
\draw (n9.input 1) -- ++(left:2mm) |- (n12.east) node[at end, above, xshift=2.0mm, yshift=-2pt]{\tiny\texttt{0000}};
\draw (n8.input 3) -- ++(left:2mm) |- (n9.output) node[at end, above, xshift=2.0mm, yshift=-2pt]{\tiny\texttt{0000}};
\node[and gate,inputs={nn}] at (-2.5,-0.16666663) (n13) {};
\node at (-4,-0.3333333) (n14) {$x_1$};
\draw (n13.input 2) -- ++(left:2mm) |- (n14.east) node[at end, above, xshift=2.0mm, yshift=-2pt]{\tiny\texttt{0000}};
\node at (-4,0.000000029802322) (n15) {$\varphi$};
\draw (n13.input 1) -- ++(left:2mm) |- (n15.east) node[at end, above, xshift=2.0mm, yshift=-2pt]{\tiny\texttt{1100}};
\draw (n8.input 2) -- ++(left:3.5mm) |- (n13.output) node[at end, above, xshift=2.0mm, yshift=-2pt]{\tiny\texttt{0000}};
\node[and gate,inputs={nn}] at (-2.5,1.1000001) (n16) {};
\node at (-4,0.93333346) (n17) {$\overline{\varphi}$};
\draw (n16.input 2) -- ++(left:2mm) |- (n17.east) node[at end, above, xshift=2.0mm, yshift=-2pt]{\tiny\texttt{0011}};
\node at (-4,1.6500002) (n18) {$\overline{x_1}$};
\draw (n16.input 1) -- ++(left:2mm) |- (n18.east) node[at end, above, xshift=2.0mm, yshift=-2pt]{\tiny\texttt{1111}};
\draw (n8.input 1) -- ++(left:2mm) |- (n16.output) node[at end, above, xshift=2.0mm, yshift=-2pt]{\tiny\texttt{0011}};
\node[circle, fill=black, inner sep=0pt, minimum size=3pt] (c0) at (-7.19,0.20000005) {};
\draw (n1.output) -- (c0) node[at start, midway, above, xshift=0.2mm, yshift=-2pt]{\tiny\texttt{1100}};
\draw (c0) |- (not.input);
\draw (not.output) -- (-4.8,-1) node[near start, midway, above, xshift=-4mm, yshift=-2pt]{\tiny\texttt{1100}};
\draw (c0) -- (-5,0.20000005);
\draw (-5,0.000000029802322) -- (n15.west);
\draw (-5,0.000000029802322) -- (-5, 0.20000005);
\draw (-4.8,0.93333346) -- (n17.west);
\draw (-4.8,-1) -- (-4.8, 0.93333346);
\draw (n8.output) -- ++(right:5mm) |- (n0.west) node[at start, midway, above, xshift=-2mm, yshift=-2pt]{\tiny\texttt{0011}};
\end{tikzpicture}\end{center}
Схема по упрощенной МКНФ:
\[f = \left(x_{1} \lor \varphi\right) \, \left(\overline{\varphi} \lor \overline{x_{1}}\right) \, \left(x_{2} \lor \overline{x_{3}} \lor \overline{x_{4}} \lor \overline{x_{5}}\right)\quad(S_Q = 18, \tau = 6)\]
\[\varphi = x_{2} \, \left(x_{4} \lor x_{3} \, x_{5}\right)\]
\begin{center}\begin{tikzpicture}[circuit logic IEC]
\node at (0,0) (n0) {$f$};
\node[and gate,inputs={nn}] at (-8,0.20000005) (n1) {};
\node[or gate,inputs={nn}] at (-9.5,0.03333336) (n2) {};
\node[and gate,inputs={nn}] at (-11,-0.13333333) (n3) {};
\node at (-12.5,-0.3) (n4) {$x_5$};
\draw (n3.input 2) -- ++(left:2mm) |- (n4.east) node[at end, above, xshift=2.0mm, yshift=-2pt]{\tiny\texttt{0101}};
\node at (-12.5,0.03333333) (n5) {$x_3$};
\draw (n3.input 1) -- ++(left:2mm) |- (n5.east) node[at end, above, xshift=2.0mm, yshift=-2pt]{\tiny\texttt{0000}};
\draw (n2.input 2) -- ++(left:2mm) |- (n3.output) node[at end, above, xshift=2.0mm, yshift=-2pt]{\tiny\texttt{0000}};
\node at (-11,0.5833334) (n6) {$x_4$};
\draw (n2.input 1) -- ++(left:2mm) |- (n6.east) node[at end, above, xshift=2.0mm, yshift=-2pt]{\tiny\texttt{0011}};
\draw (n1.input 2) -- ++(left:2mm) |- (n2.output) node[at end, above, xshift=2.0mm, yshift=-2pt]{\tiny\texttt{0011}};
\node at (-9.5,0.91666675) (n7) {$x_2$};
\draw (n1.input 1) -- ++(left:2mm) |- (n7.east) node[at end, above, xshift=2.0mm, yshift=-2pt]{\tiny\texttt{0011}};
\node[not gate] at (-6.5,-1) (not) {};
\node[and gate,inputs={nnn}] at (-1,0) (n8) {};
\node[or gate,inputs={nnnn}] at (-2.5,-1.2666667) (n9) {};
\node at (-4,-1.7666668) (n10) {$\overline{x_5}$};
\draw (n9.input 4) -- ++(left:2mm) |- (n10.east) node[at end, above, xshift=2.0mm, yshift=-2pt]{\tiny\texttt{1010}};
\node at (-4,-1.4333334) (n11) {$\overline{x_4}$};
\draw (n9.input 3) -- ++(left:3.5mm) |- (n11.east) node[at end, above, xshift=2.0mm, yshift=-2pt]{\tiny\texttt{1100}};
\node at (-4,-1.1) (n12) {$\overline{x_3}$};
\draw (n9.input 2) -- ++(left:3.5mm) |- (n12.east) node[at end, above, xshift=2.0mm, yshift=-2pt]{\tiny\texttt{1111}};
\node at (-4,-0.76666665) (n13) {$x_2$};
\draw (n9.input 1) -- ++(left:2mm) |- (n13.east) node[at end, above, xshift=2.0mm, yshift=-2pt]{\tiny\texttt{0011}};
\draw (n8.input 3) -- ++(left:2mm) |- (n9.output) node[at end, above, xshift=2.0mm, yshift=-2pt]{\tiny\texttt{1111}};
\node[or gate,inputs={nn}] at (-2.5,0.116666794) (n14) {};
\node at (-4,-0.43333322) (n15) {$\overline{x_1}$};
\draw (n14.input 2) -- ++(left:2mm) |- (n15.east) node[at end, above, xshift=2.0mm, yshift=-2pt]{\tiny\texttt{1111}};
\node at (-4,0.28333345) (n16) {$\overline{\varphi}$};
\draw (n14.input 1) -- ++(left:2mm) |- (n16.east) node[at end, above, xshift=2.0mm, yshift=-2pt]{\tiny\texttt{1100}};
\draw (n8.input 2) -- ++(left:3.5mm) |- (n14.output) node[at end, above, xshift=2.0mm, yshift=-2pt]{\tiny\texttt{1111}};
\node[or gate,inputs={nn}] at (-2.5,1.3833334) (n17) {};
\node at (-4,1.2166668) (n18) {$\varphi$};
\draw (n17.input 2) -- ++(left:2mm) |- (n18.east) node[at end, above, xshift=2.0mm, yshift=-2pt]{\tiny\texttt{0011}};
\node at (-4,1.5500002) (n19) {$x_1$};
\draw (n17.input 1) -- ++(left:2mm) |- (n19.east) node[at end, above, xshift=2.0mm, yshift=-2pt]{\tiny\texttt{0000}};
\draw (n8.input 1) -- ++(left:2mm) |- (n17.output) node[at end, above, xshift=2.0mm, yshift=-2pt]{\tiny\texttt{0011}};
\node[circle, fill=black, inner sep=0pt, minimum size=3pt] (c0) at (-7.19,0.20000005) {};
\draw (n1.output) -- (c0) node[at start, midway, above, xshift=0.2mm, yshift=-2pt]{\tiny\texttt{0011}};
\draw (c0) |- (not.input);
\draw (not.output) -- (-4.8,-1) node[near start, midway, above, xshift=-4mm, yshift=-2pt]{\tiny\texttt{0011}};
\draw (c0) -- (-5,0.20000005);
\draw (-5,1.2166668) -- (n18.west);
\draw (-5,0.20000005) -- (-5, 1.2166668);
\draw (-4.8,0.28333345) -- (n16.west);
\draw (-4.8,-1) -- (-4.8, 0.28333345);
\draw (n8.output) -- ++(right:5mm) |- (n0.west) node[at start, midway, above, xshift=-2mm, yshift=-2pt]{\tiny\texttt{0011}};
\end{tikzpicture}\end{center}
\subsection*{Сокращенный булев базис (И, НЕ)}
Схема по упрощенной МДНФ в базисе И, НЕ:
\[f = \overline{\overline{\overline{x_{1}} \, \overline{\varphi}} \, \overline{\varphi \, x_{1}} \, \overline{x_{1} \, \overline{x_{2}} \, \overline{x_{4}}}}\quad(S_Q = 23, \tau = 8)\]
\[\varphi = \overline{x_{2} \, x_{4}} \, \overline{x_{3} \, x_{5}}\]
\begin{center}\begin{tikzpicture}[circuit logic IEC]
\node at (0,0) (n0) {$f$};
\node[and gate,inputs={nn}] at (-11,0.20000005) (n1) {};
\node[and gate,inputs={nn}] at (-14,-0.3499999) (n3) {};
\node at (-15.5,-0.5166666) (n4) {$x_5$};
\draw (n3.input 2) -- ++(left:2mm) |- (n4.east) node[at end, above, xshift=2.0mm, yshift=-2pt]{\tiny\texttt{0101}};
\node at (-15.5,-0.18333325) (n5) {$x_3$};
\draw (n3.input 1) -- ++(left:2mm) |- (n5.east) node[at end, above, xshift=2.0mm, yshift=-2pt]{\tiny\texttt{0000}};
\node[not gate] at (-12.5,-0.3499999) (n2) {};
\draw (n3.output) -- ++(right:3mm) |- (n2.west) node[at start, above, xshift=-0.3mm, yshift=-2pt]{\tiny\texttt{0000}};
\draw (n1.input 2) -- ++(left:2mm) |- (n2.output) node[at end, above, xshift=2.0mm, yshift=-2pt]{\tiny\texttt{1111}};
\node[and gate,inputs={nn}] at (-14,0.7500001) (n7) {};
\node at (-15.5,0.58333343) (n8) {$x_4$};
\draw (n7.input 2) -- ++(left:2mm) |- (n8.east) node[at end, above, xshift=2.0mm, yshift=-2pt]{\tiny\texttt{0011}};
\node at (-15.5,0.91666675) (n9) {$x_2$};
\draw (n7.input 1) -- ++(left:2mm) |- (n9.east) node[at end, above, xshift=2.0mm, yshift=-2pt]{\tiny\texttt{0011}};
\node[not gate] at (-12.5,0.7500001) (n6) {};
\draw (n7.output) -- ++(right:3mm) |- (n6.west) node[at start, above, xshift=-0.3mm, yshift=-2pt]{\tiny\texttt{0011}};
\draw (n1.input 1) -- ++(left:2mm) |- (n6.output) node[at end, above, xshift=2.0mm, yshift=-2pt]{\tiny\texttt{1100}};
\node[not gate] at (-9.5,-1) (not) {};
\node[and gate,inputs={nnn}] at (-2.5,0) (n11) {};
\node[and gate,inputs={nnn}] at (-5.5,-1.2666667) (n13) {};
\node at (-7,-1.6) (n14) {$\overline{x_4}$};
\draw (n13.input 3) -- ++(left:2mm) |- (n14.east) node[at end, above, xshift=2.0mm, yshift=-2pt]{\tiny\texttt{1100}};
\node at (-7,-1.2666667) (n15) {$\overline{x_2}$};
\draw (n13.input 2) -- ++(left:3.5mm) |- (n15.east) node[at end, above, xshift=2.0mm, yshift=-2pt]{\tiny\texttt{1100}};
\node at (-7,-0.9333333) (n16) {$x_1$};
\draw (n13.input 1) -- ++(left:2mm) |- (n16.east) node[at end, above, xshift=2.0mm, yshift=-2pt]{\tiny\texttt{0000}};
\node[not gate] at (-4,-1.2666667) (n12) {};
\draw (n13.output) -- ++(right:3mm) |- (n12.west) node[at start, above, xshift=-0.3mm, yshift=-2pt]{\tiny\texttt{0000}};
\draw (n11.input 3) -- ++(left:2mm) |- (n12.output) node[at end, above, xshift=2.0mm, yshift=-2pt]{\tiny\texttt{1111}};
\node[and gate,inputs={nn}] at (-5.5,-0.16666663) (n18) {};
\node at (-7,-0.3333333) (n19) {$x_1$};
\draw (n18.input 2) -- ++(left:2mm) |- (n19.east) node[at end, above, xshift=2.0mm, yshift=-2pt]{\tiny\texttt{0000}};
\node at (-7,0.000000029802322) (n20) {$\varphi$};
\draw (n18.input 1) -- ++(left:2mm) |- (n20.east) node[at end, above, xshift=2.0mm, yshift=-2pt]{\tiny\texttt{1100}};
\node[not gate] at (-4,-0.16666663) (n17) {};
\draw (n18.output) -- ++(right:3mm) |- (n17.west) node[at start, above, xshift=-0.3mm, yshift=-2pt]{\tiny\texttt{0000}};
\draw (n11.input 2) -- ++(left:3.5mm) |- (n17.output) node[at end, above, xshift=2.0mm, yshift=-2pt]{\tiny\texttt{1111}};
\node[and gate,inputs={nn}] at (-5.5,1.1000001) (n22) {};
\node at (-7,0.93333346) (n23) {$\overline{\varphi}$};
\draw (n22.input 2) -- ++(left:2mm) |- (n23.east) node[at end, above, xshift=2.0mm, yshift=-2pt]{\tiny\texttt{0011}};
\node at (-7,1.6500002) (n24) {$\overline{x_1}$};
\draw (n22.input 1) -- ++(left:2mm) |- (n24.east) node[at end, above, xshift=2.0mm, yshift=-2pt]{\tiny\texttt{1111}};
\node[not gate] at (-4,1.1000001) (n21) {};
\draw (n22.output) -- ++(right:3mm) |- (n21.west) node[at start, above, xshift=-0.3mm, yshift=-2pt]{\tiny\texttt{0011}};
\draw (n11.input 1) -- ++(left:2mm) |- (n21.output) node[at end, above, xshift=2.0mm, yshift=-2pt]{\tiny\texttt{1100}};
\node[not gate] at (-1,0) (n10) {};
\draw (n11.output) -- ++(right:3mm) |- (n10.west) node[at start, above, xshift=-0.3mm, yshift=-2pt]{\tiny\texttt{1100}};
\node[circle, fill=black, inner sep=0pt, minimum size=3pt] (c0) at (-10.19,0.20000005) {};
\draw (n1.output) -- (c0) node[at start, midway, above, xshift=0.2mm, yshift=-2pt]{\tiny\texttt{1100}};
\draw (c0) |- (not.input);
\draw (not.output) -- (-7.8,-1) node[near start, midway, above, xshift=-4mm, yshift=-2pt]{\tiny\texttt{1100}};
\draw (c0) -- (-8,0.20000005);
\draw (-8,0.000000029802322) -- (n20.west);
\draw (-8,0.000000029802322) -- (-8, 0.20000005);
\draw (-7.8,0.93333346) -- (n23.west);
\draw (-7.8,-1) -- (-7.8, 0.93333346);
\draw (n10.output) -- ++(right:5mm) |- (n0.west) node[at start, midway, above, xshift=-2mm, yshift=-2pt]{\tiny\texttt{0011}};
\end{tikzpicture}\end{center}
\newpage
Схема по упрощенной МКНФ в базисе И, НЕ:
\[f = \overline{\overline{x_{1}} \, \overline{\varphi}} \, \overline{\varphi \, x_{1}} \, \overline{\overline{x_{2}} \, x_{3} \, x_{4} \, x_{5}}\quad(S_Q = 23, \tau = 9)\]
\[\varphi = x_{2} \, \overline{\overline{x_{4}} \, \overline{x_{3} \, x_{5}}}\]
\begin{center}\begin{tikzpicture}[circuit logic IEC]
\node at (0,0) (n0) {$f$};
\node[and gate,inputs={nn}] at (-9.5,0.20000005) (n1) {};
\node[and gate,inputs={nn}] at (-12.5,0.03333336) (n3) {};
\node[and gate,inputs={nn}] at (-15.5,-0.13333333) (n5) {};
\node at (-17,-0.3) (n6) {$x_5$};
\draw (n5.input 2) -- ++(left:2mm) |- (n6.east) node[at end, above, xshift=2.0mm, yshift=-2pt]{\tiny\texttt{0101}};
\node at (-17,0.03333333) (n7) {$x_3$};
\draw (n5.input 1) -- ++(left:2mm) |- (n7.east) node[at end, above, xshift=2.0mm, yshift=-2pt]{\tiny\texttt{0000}};
\node[not gate] at (-14,-0.13333333) (n4) {};
\draw (n5.output) -- ++(right:3mm) |- (n4.west) node[at start, above, xshift=-0.3mm, yshift=-2pt]{\tiny\texttt{0000}};
\draw (n3.input 2) -- ++(left:2mm) |- (n4.output) node[at end, above, xshift=2.0mm, yshift=-2pt]{\tiny\texttt{1111}};
\node at (-14,0.5833334) (n8) {$\overline{x_4}$};
\draw (n3.input 1) -- ++(left:2mm) |- (n8.east) node[at end, above, xshift=2.0mm, yshift=-2pt]{\tiny\texttt{1100}};
\node[not gate] at (-11,0.03333336) (n2) {};
\draw (n3.output) -- ++(right:3mm) |- (n2.west) node[at start, above, xshift=-0.3mm, yshift=-2pt]{\tiny\texttt{1100}};
\draw (n1.input 2) -- ++(left:2mm) |- (n2.output) node[at end, above, xshift=2.0mm, yshift=-2pt]{\tiny\texttt{0011}};
\node at (-11,0.91666675) (n9) {$x_2$};
\draw (n1.input 1) -- ++(left:2mm) |- (n9.east) node[at end, above, xshift=2.0mm, yshift=-2pt]{\tiny\texttt{0011}};
\node[not gate] at (-8,-1) (not) {};
\node[and gate,inputs={nnn}] at (-1,0) (n10) {};
\node[and gate,inputs={nnnn}] at (-4,-1.2666667) (n12) {};
\node at (-5.5,-1.7666668) (n13) {$x_5$};
\draw (n12.input 4) -- ++(left:2mm) |- (n13.east) node[at end, above, xshift=2.0mm, yshift=-2pt]{\tiny\texttt{0101}};
\node at (-5.5,-1.4333334) (n14) {$x_4$};
\draw (n12.input 3) -- ++(left:3.5mm) |- (n14.east) node[at end, above, xshift=2.0mm, yshift=-2pt]{\tiny\texttt{0011}};
\node at (-5.5,-1.1) (n15) {$x_3$};
\draw (n12.input 2) -- ++(left:3.5mm) |- (n15.east) node[at end, above, xshift=2.0mm, yshift=-2pt]{\tiny\texttt{0000}};
\node at (-5.5,-0.76666665) (n16) {$\overline{x_2}$};
\draw (n12.input 1) -- ++(left:2mm) |- (n16.east) node[at end, above, xshift=2.0mm, yshift=-2pt]{\tiny\texttt{1100}};
\node[not gate] at (-2.5,-1.2666667) (n11) {};
\draw (n12.output) -- ++(right:3mm) |- (n11.west) node[at start, above, xshift=-0.3mm, yshift=-2pt]{\tiny\texttt{0000}};
\draw (n10.input 3) -- ++(left:2mm) |- (n11.output) node[at end, above, xshift=2.0mm, yshift=-2pt]{\tiny\texttt{1111}};
\node[and gate,inputs={nn}] at (-4,-0.049999893) (n18) {};
\node at (-5.5,-0.21666658) (n19) {$x_1$};
\draw (n18.input 2) -- ++(left:2mm) |- (n19.east) node[at end, above, xshift=2.0mm, yshift=-2pt]{\tiny\texttt{0000}};
\node at (-5.5,0.116666764) (n20) {$\varphi$};
\draw (n18.input 1) -- ++(left:2mm) |- (n20.east) node[at end, above, xshift=2.0mm, yshift=-2pt]{\tiny\texttt{0011}};
\node[not gate] at (-2.5,-0.049999893) (n17) {};
\draw (n18.output) -- ++(right:3mm) |- (n17.west) node[at start, above, xshift=-0.3mm, yshift=-2pt]{\tiny\texttt{0000}};
\draw (n10.input 2) -- ++(left:3.5mm) |- (n17.output) node[at end, above, xshift=2.0mm, yshift=-2pt]{\tiny\texttt{1111}};
\node[and gate,inputs={nn}] at (-4,1.2166668) (n22) {};
\node at (-5.5,1.0500002) (n23) {$\overline{\varphi}$};
\draw (n22.input 2) -- ++(left:2mm) |- (n23.east) node[at end, above, xshift=2.0mm, yshift=-2pt]{\tiny\texttt{1100}};
\node at (-5.5,1.7666669) (n24) {$\overline{x_1}$};
\draw (n22.input 1) -- ++(left:2mm) |- (n24.east) node[at end, above, xshift=2.0mm, yshift=-2pt]{\tiny\texttt{1111}};
\node[not gate] at (-2.5,1.2166668) (n21) {};
\draw (n22.output) -- ++(right:3mm) |- (n21.west) node[at start, above, xshift=-0.3mm, yshift=-2pt]{\tiny\texttt{1100}};
\draw (n10.input 1) -- ++(left:2mm) |- (n21.output) node[at end, above, xshift=2.0mm, yshift=-2pt]{\tiny\texttt{0011}};
\node[circle, fill=black, inner sep=0pt, minimum size=3pt] (c0) at (-8.69,0.20000005) {};
\draw (n1.output) -- (c0) node[at start, midway, above, xshift=0.2mm, yshift=-2pt]{\tiny\texttt{0011}};
\draw (c0) |- (not.input);
\draw (not.output) -- (-6.3,-1) node[near start, midway, above, xshift=-4mm, yshift=-2pt]{\tiny\texttt{0011}};
\draw (c0) -- (-6.5,0.20000005);
\draw (-6.5,0.116666764) -- (n20.west);
\draw (-6.5,0.116666764) -- (-6.5, 0.20000005);
\draw (-6.3,1.0500002) -- (n23.west);
\draw (-6.3,-1) -- (-6.3, 1.0500002);
\draw (n10.output) -- ++(right:5mm) |- (n0.west) node[at start, midway, above, xshift=-2mm, yshift=-2pt]{\tiny\texttt{0011}};
\end{tikzpicture}\end{center}
\subsection*{Универсальный базис (И-НЕ, 2 входа)}
Схема по упрощенной МДНФ в базисе И-НЕ с ограничением на число входов:
\[f = \overline{\overline{\overline{x_{1}} \, \varphi} \, \overline{\overline{\overline{\overline{\varphi} \, x_{1}} \, \overline{x_{1} \, \overline{\overline{\overline{x_{2}} \, \overline{x_{4}}}}}}}}\quad(S_Q = 24, \tau = 7)\]
\[\varphi = \overline{\overline{x_{3} \, x_{5}} \, \overline{x_{2} \, x_{4}}}\]
\begin{center}\begin{tikzpicture}[circuit logic IEC]
\node at (0,0) (n0) {$f$};
\node[nand gate,inputs={nn}] at (-14,0.20000005) (n1) {};
\node[nand gate,inputs={nn}] at (-15.5,-0.3499999) (n2) {};
\node at (-17,-0.5166666) (n3) {$x_4$};
\draw (n2.input 2) -- ++(left:2mm) |- (n3.east) node[at end, above, xshift=2.0mm, yshift=-2pt]{\tiny\texttt{0011}};
\node at (-17,-0.18333325) (n4) {$x_2$};
\draw (n2.input 1) -- ++(left:2mm) |- (n4.east) node[at end, above, xshift=2.0mm, yshift=-2pt]{\tiny\texttt{0011}};
\draw (n1.input 2) -- ++(left:2mm) |- (n2.output) node[at end, above, xshift=2.0mm, yshift=-2pt]{\tiny\texttt{1100}};
\node[nand gate,inputs={nn}] at (-15.5,0.7500001) (n5) {};
\node at (-17,0.58333343) (n6) {$x_5$};
\draw (n5.input 2) -- ++(left:2mm) |- (n6.east) node[at end, above, xshift=2.0mm, yshift=-2pt]{\tiny\texttt{0101}};
\node at (-17,0.91666675) (n7) {$x_3$};
\draw (n5.input 1) -- ++(left:2mm) |- (n7.east) node[at end, above, xshift=2.0mm, yshift=-2pt]{\tiny\texttt{0000}};
\draw (n1.input 1) -- ++(left:2mm) |- (n5.output) node[at end, above, xshift=2.0mm, yshift=-2pt]{\tiny\texttt{1111}};
\node[nand gate] at (-12.5,-1) (not) {};
\node[nand gate,inputs={nn}] at (-1,0) (n8) {};
\node[nand gate,inputs={nn}] at (-4,-0.54999995) (n10) {};
\node[nand gate,inputs={nn}] at (-5.5,-1.2666667) (n11) {};
\node[nand gate,inputs={nn}] at (-8.5,-1.4333334) (n13) {};
\node at (-10,-1.6) (n14) {$\overline{x_4}$};
\draw (n13.input 2) -- ++(left:2mm) |- (n14.east) node[at end, above, xshift=2.0mm, yshift=-2pt]{\tiny\texttt{1100}};
\node at (-10,-1.2666667) (n15) {$\overline{x_2}$};
\draw (n13.input 1) -- ++(left:2mm) |- (n15.east) node[at end, above, xshift=2.0mm, yshift=-2pt]{\tiny\texttt{1100}};
\node[nand gate] at (-7,-1.4333334) (n12) {};
\node[circle, fill=black, inner sep=0pt, minimum size=3pt] (n16) at (-7.5,-1.4333334) {};
\draw (n16) |- (n12.input 1);
\draw (n16) |- (n12.input 2);
\draw (n13.output) -- ++(right:3mm) |- (n16) node[at start, above, xshift=-0.3mm, yshift=-2pt]{\tiny\texttt{0011}};
\draw (n11.input 2) -- ++(left:2mm) |- (n12.output) node[at end, above, xshift=2.0mm, yshift=-2pt]{\tiny\texttt{1100}};
\node at (-7,-0.7166667) (n17) {$x_1$};
\draw (n11.input 1) -- ++(left:2mm) |- (n17.east) node[at end, above, xshift=2.0mm, yshift=-2pt]{\tiny\texttt{0000}};
\draw (n10.input 2) -- ++(left:2mm) |- (n11.output) node[at end, above, xshift=2.0mm, yshift=-2pt]{\tiny\texttt{1111}};
\node[nand gate,inputs={nn}] at (-5.5,0.16666675) (n18) {};
\node at (-7,-0.38333327) (n19) {$x_1$};
\draw (n18.input 2) -- ++(left:2mm) |- (n19.east) node[at end, above, xshift=2.0mm, yshift=-2pt]{\tiny\texttt{0000}};
\node at (-7,0.3333334) (n20) {$\overline{\varphi}$};
\draw (n18.input 1) -- ++(left:2mm) |- (n20.east) node[at end, above, xshift=2.0mm, yshift=-2pt]{\tiny\texttt{1100}};
\draw (n10.input 1) -- ++(left:2mm) |- (n18.output) node[at end, above, xshift=2.0mm, yshift=-2pt]{\tiny\texttt{1111}};
\node[nand gate] at (-2.5,-0.54999995) (n9) {};
\node[circle, fill=black, inner sep=0pt, minimum size=3pt] (n21) at (-3,-0.54999995) {};
\draw (n21) |- (n9.input 1);
\draw (n21) |- (n9.input 2);
\draw (n10.output) -- ++(right:3mm) |- (n21) node[at start, above, xshift=-0.3mm, yshift=-2pt]{\tiny\texttt{0000}};
\draw (n8.input 2) -- ++(left:2mm) |- (n9.output) node[at end, above, xshift=2.0mm, yshift=-2pt]{\tiny\texttt{1111}};
\node[nand gate,inputs={nn}] at (-2.5,1.4333335) (n22) {};
\node at (-4,1.2666668) (n23) {$\varphi$};
\draw (n22.input 2) -- ++(left:2mm) |- (n23.east) node[at end, above, xshift=2.0mm, yshift=-2pt]{\tiny\texttt{0011}};
\node at (-4,1.6000001) (n24) {$\overline{x_1}$};
\draw (n22.input 1) -- ++(left:2mm) |- (n24.east) node[at end, above, xshift=2.0mm, yshift=-2pt]{\tiny\texttt{1111}};
\draw (n8.input 1) -- ++(left:2mm) |- (n22.output) node[at end, above, xshift=2.0mm, yshift=-2pt]{\tiny\texttt{1100}};
\node[circle, fill=black, inner sep=0pt, minimum size=3pt] (c0) at (-13.19,0.20000005) {};
\draw (n1.output) -- (c0) node[at start, midway, above, xshift=0.2mm, yshift=-2pt]{\tiny\texttt{0011}};
\node[circle, fill=black, inner sep=0pt, minimum size=3pt] (c1) at (-13,-1) {};
\draw (c0) |- (c1);
\draw (c1) |- (not.input 1);
\draw (c1) |- (not.input 2);
\draw (not.output) -- (-10.8,-1) node[near start, midway, above, xshift=-4mm, yshift=-2pt]{\tiny\texttt{0011}};
\draw (c0) -- (-11,0.20000005);
\draw (-11,1.2666668) -- (n23.west);
\draw (-11,0.20000005) -- (-11, 1.2666668);
\draw (-10.8,0.3333334) -- (n20.west);
\draw (-10.8,-1) -- (-10.8, 0.3333334);
\draw (n8.output) -- ++(right:5mm) |- (n0.west) node[at start, midway, above, xshift=-2mm, yshift=-2pt]{\tiny\texttt{0011}};
\end{tikzpicture}\end{center}
Схема по упрощенной МКНФ в базисе И-НЕ с ограничением на число входов:
\[f = \overline{\overline{\overline{\overline{x_{1}} \, \varphi} \, \overline{\overline{\overline{\overline{\varphi} \, x_{1}} \, \overline{\overline{\overline{\overline{x_{2}} \, x_{3}}} \, \overline{\overline{x_{4} \, x_{5}}}}}}}}\quad(S_Q = 30, \tau = 9)\]
\[\varphi = \overline{x_{2} \, \overline{\overline{x_{4}} \, \overline{x_{3} \, x_{5}}}}\]
\begin{center}\scalebox{0.9}{\begin{tikzpicture}[circuit logic IEC]
\node at (0,0) (n0) {$f$};
\node[nand gate,inputs={nn}] at (-15.5,0.20000005) (n1) {};
\node[nand gate,inputs={nn}] at (-17,0.03333336) (n2) {};
\node[nand gate,inputs={nn}] at (-18.5,-0.13333333) (n3) {};
\node at (-20,-0.3) (n4) {$x_5$};
\draw (n3.input 2) -- ++(left:2mm) |- (n4.east) node[at end, above, xshift=2.0mm, yshift=-2pt]{\tiny\texttt{0101}};
\node at (-20,0.03333333) (n5) {$x_3$};
\draw (n3.input 1) -- ++(left:2mm) |- (n5.east) node[at end, above, xshift=2.0mm, yshift=-2pt]{\tiny\texttt{0000}};
\draw (n2.input 2) -- ++(left:2mm) |- (n3.output) node[at end, above, xshift=2.0mm, yshift=-2pt]{\tiny\texttt{1111}};
\node at (-18.5,0.5833334) (n6) {$\overline{x_4}$};
\draw (n2.input 1) -- ++(left:2mm) |- (n6.east) node[at end, above, xshift=2.0mm, yshift=-2pt]{\tiny\texttt{1100}};
\draw (n1.input 2) -- ++(left:2mm) |- (n2.output) node[at end, above, xshift=2.0mm, yshift=-2pt]{\tiny\texttt{0011}};
\node at (-17,0.91666675) (n7) {$x_2$};
\draw (n1.input 1) -- ++(left:2mm) |- (n7.east) node[at end, above, xshift=2.0mm, yshift=-2pt]{\tiny\texttt{0011}};
\node[nand gate] at (-14,-1) (not) {};
\node[nand gate,inputs={nn}] at (-2.5,0) (n9) {};
\node[nand gate,inputs={nn}] at (-5.5,-0.54999995) (n11) {};
\node[nand gate,inputs={nn}] at (-7,-1.2666667) (n12) {};
\node[nand gate,inputs={nn}] at (-10,-1.8166666) (n14) {};
\node at (-11.5,-1.9833332) (n15) {$x_5$};
\draw (n14.input 2) -- ++(left:2mm) |- (n15.east) node[at end, above, xshift=2.0mm, yshift=-2pt]{\tiny\texttt{0101}};
\node at (-11.5,-1.6499999) (n16) {$x_4$};
\draw (n14.input 1) -- ++(left:2mm) |- (n16.east) node[at end, above, xshift=2.0mm, yshift=-2pt]{\tiny\texttt{0011}};
\node[nand gate] at (-8.5,-1.8166666) (n13) {};
\node[circle, fill=black, inner sep=0pt, minimum size=3pt] (n17) at (-9,-1.8166666) {};
\draw (n17) |- (n13.input 1);
\draw (n17) |- (n13.input 2);
\draw (n14.output) -- ++(right:3mm) |- (n17) node[at start, above, xshift=-0.3mm, yshift=-2pt]{\tiny\texttt{1110}};
\draw (n12.input 2) -- ++(left:2mm) |- (n13.output) node[at end, above, xshift=2.0mm, yshift=-2pt]{\tiny\texttt{0001}};
\node[nand gate,inputs={nn}] at (-10,-0.7166666) (n19) {};
\node at (-11.5,-0.88333327) (n20) {$x_3$};
\draw (n19.input 2) -- ++(left:2mm) |- (n20.east) node[at end, above, xshift=2.0mm, yshift=-2pt]{\tiny\texttt{0000}};
\node at (-11.5,-0.54999995) (n21) {$\overline{x_2}$};
\draw (n19.input 1) -- ++(left:2mm) |- (n21.east) node[at end, above, xshift=2.0mm, yshift=-2pt]{\tiny\texttt{1100}};
\node[nand gate] at (-8.5,-0.7166666) (n18) {};
\node[circle, fill=black, inner sep=0pt, minimum size=3pt] (n22) at (-9,-0.7166666) {};
\draw (n22) |- (n18.input 1);
\draw (n22) |- (n18.input 2);
\draw (n19.output) -- ++(right:3mm) |- (n22) node[at start, above, xshift=-0.3mm, yshift=-2pt]{\tiny\texttt{1111}};
\draw (n12.input 1) -- ++(left:2mm) |- (n18.output) node[at end, above, xshift=2.0mm, yshift=-2pt]{\tiny\texttt{0000}};
\draw (n11.input 2) -- ++(left:2mm) |- (n12.output) node[at end, above, xshift=2.0mm, yshift=-2pt]{\tiny\texttt{1111}};
\node[nand gate,inputs={nn}] at (-7,0.55) (n23) {};
\node at (-8.5,0) (n24) {$x_1$};
\draw (n23.input 2) -- ++(left:2mm) |- (n24.east) node[at end, above, xshift=2.0mm, yshift=-2pt]{\tiny\texttt{0000}};
\node at (-8.5,0.7166667) (n25) {$\overline{\varphi}$};
\draw (n23.input 1) -- ++(left:2mm) |- (n25.east) node[at end, above, xshift=2.0mm, yshift=-2pt]{\tiny\texttt{0011}};
\draw (n11.input 1) -- ++(left:2mm) |- (n23.output) node[at end, above, xshift=2.0mm, yshift=-2pt]{\tiny\texttt{1111}};
\node[nand gate] at (-4,-0.54999995) (n10) {};
\node[circle, fill=black, inner sep=0pt, minimum size=3pt] (n26) at (-4.5,-0.54999995) {};
\draw (n26) |- (n10.input 1);
\draw (n26) |- (n10.input 2);
\draw (n11.output) -- ++(right:3mm) |- (n26) node[at start, above, xshift=-0.3mm, yshift=-2pt]{\tiny\texttt{0000}};
\draw (n9.input 2) -- ++(left:2mm) |- (n10.output) node[at end, above, xshift=2.0mm, yshift=-2pt]{\tiny\texttt{1111}};
\node[nand gate,inputs={nn}] at (-4,1.8166667) (n27) {};
\node at (-5.5,1.65) (n28) {$\varphi$};
\draw (n27.input 2) -- ++(left:2mm) |- (n28.east) node[at end, above, xshift=2.0mm, yshift=-2pt]{\tiny\texttt{1100}};
\node at (-5.5,1.9833333) (n29) {$\overline{x_1}$};
\draw (n27.input 1) -- ++(left:2mm) |- (n29.east) node[at end, above, xshift=2.0mm, yshift=-2pt]{\tiny\texttt{1111}};
\draw (n9.input 1) -- ++(left:2mm) |- (n27.output) node[at end, above, xshift=2.0mm, yshift=-2pt]{\tiny\texttt{0011}};
\node[nand gate] at (-1,0) (n8) {};
\node[circle, fill=black, inner sep=0pt, minimum size=3pt] (n30) at (-1.5,0) {};
\draw (n30) |- (n8.input 1);
\draw (n30) |- (n8.input 2);
\draw (n9.output) -- ++(right:3mm) |- (n30) node[at start, above, xshift=-0.3mm, yshift=-2pt]{\tiny\texttt{1100}};
\node[circle, fill=black, inner sep=0pt, minimum size=3pt] (c0) at (-14.69,0.20000005) {};
\draw (n1.output) -- (c0) node[at start, midway, above, xshift=0.2mm, yshift=-2pt]{\tiny\texttt{1100}};
\node[circle, fill=black, inner sep=0pt, minimum size=3pt] (c1) at (-14.5,-1) {};
\draw (c0) |- (c1);
\draw (c1) |- (not.input 1);
\draw (c1) |- (not.input 2);
\draw (not.output) -- (-12.3,-1) node[near start, midway, above, xshift=-4mm, yshift=-2pt]{\tiny\texttt{1100}};
\draw (c0) -- (-12.5,0.20000005);
\draw (-12.5,1.65) -- (n28.west);
\draw (-12.5,0.20000005) -- (-12.5, 1.65);
\draw (-12.3,0.7166667) -- (n25.west);
\draw (-12.3,-1) -- (-12.3, 0.7166667);
\draw (n8.output) -- ++(right:5mm) |- (n0.west) node[at start, midway, above, xshift=-2mm, yshift=-2pt]{\tiny\texttt{0011}};
\end{tikzpicture}}\end{center}

\end{document}


